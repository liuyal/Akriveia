
\setcounter{section}{0}
\section{Introduction}
\bigskip

\subsection{Background}
Over the last couple of decades urban centers around the world have faced substantial population growth. As a result, the number of larger and more complex buildings in these areas around the world is rapidly increasing. With a large population and massive buildings in relatively small areas, in the event of a disaster there can be a much higher risk for damage and casualties. Due to increased urbanization and complexity of urban structures, search and rescue operations in indoor urban environments face various complications and uncertainties. According to Statistics Canada an average of 135 fire related deaths occur each year from 2010 to 2014 \cite{R1}.

\bigskip
In current practices, emergency management know little about the building until they arrive at the scene; where they have to quickly evaluate new information and take actions appropriate to the situation \cite{R2}. Once responders are on scene, assessments of the structure are conducted with readily available blueprints of buildings. Situational data are created dynamically during this process and the actual rescue process heavily depends on the situational awareness of the first line of emergency response operators \cite{R3}.

\bigskip
An important issue that must be considered is how emergency first responders should be dispatched inside the building in the event of a disaster in order to minimize search and rescue time. In order to pinpoint locations of trapped victims quickly and accurately it is critical to have precise data. Proper emergency planning and organization takes a substantial amount of time, and having additional information on the locations of trapped, incapacitated or immobile personnel would improve first responders’ situational awareness which would then improve their own safety and possibly reduce fatality rates.

\bigskip
As such, the need for a distinct indoor positioning rescue system is crucial in getting fast and reliable information that allows the first responders to be dispatched in such a way that the most optimal route is taken during the search and rescue operations. The Akriveia Beacon by TRIWAVE SYSTEMS focuses on improving the locating and rescue process of personnel trapped in buildings during or after small scale disasters such as fires and low magnitude earthquakes. This is done through a system of Ultra Wide-Band (\Gls{UWB}) Beacons and \Gls{ID} tags for accurate, near real-time location of personnel.

\bigskip
Ultra-Wideband radio modules are small radio transceivers using ultra-wide band radio spectrum to communicate with one another. Each ID tag uses a UWB transceiver module to communicate with the beacon system. Given the time between sending and receiving transmission data, the distance can be estimated via \Gls{RSSI} or time of flight. The Beacons will then forward these distance estimations to a data processing unit for trilateration calculation, where it will use trilateration to calculate the near real time location of each individual ID tag. This system design allows for multiple ID tags as well as more than three anchor beacons to provide more accuracy through redundancy, making it modular and extendable.

\break

\subsection{Scope}
This requirement specification document outlines functional and non-functional requirements expected of the Akriveia Beacon product through three different phases of development as shown in Figure \ref{dev}. The three different phases including: the proof-of-concept phase, prototype phase, and final product phase.
\medskip

\begin{figure}[H]
\centering
    \includegraphics[scale=0.4]{./images/dev-path.png}
    \caption{Development Cycle}
    \label{dev}
\end{figure}

A high level design of the system hardware and software is also presented to demonstrate the overall system architecture and functionality of the product. The requirement section of this document is divided into five main categories as shown in the numbered list below. These requirements will indicate the constraints, demands, necessities, needs, and parameters that must be met or satisfied within the project time frame.

\begin{enumerate}
\setlength\itemsep{0.25em}
	\item General requirement 
	\item Hardware requirement 
	\item Electrical requirement 
	\item Software requirement 
	\item Performance requirement 
	\item Safety Requirements
	\item Sustainability  Requirements
\end{enumerate}
\medskip
Furthermore, any relevant engineering standards, safety and sustainability practices are illustrated to demonstrate how the engineers at TRIWAVE SYSTEMS will ensure that the final product is both economically and environmentally safe and sustainable. By following strict engineering standards and proper engineering practices TRIWAVE SYSTEMS can guarantee that the final product will use materials, processes and services fit for their intended purposes.

\bigskip
\subsection{Intended Audience}
This document is presented by engineers at TRIWAVE SYSTEMS as a guide for the functional and nonfunctional requirements of the Akriveia Beacon product. The intended audience of this document includes but is not limited to, potential clients and/or partners, the supervising professors Dr. Craig Scratchley and Dr. Andrew Rawicz, associated teaching assistants and fellow TRIWAVE SYSTEMS members. The hardware and software engineers of the project can reference this document during the development and/or testing stages of the project for clarification.

\break

\subsection{Requirement Classification}
For consistency purposes, the following requirement classification code convention is used to describe and organize requirements listed throughout this document. 
\medskip
\begin{center}
	\textbf{[ REQ.SE.\# - X ]} 
\end{center}

\bgroup
\def\arraystretch{1.5}
\begin{table}[H]
\centering
\begin{tabular}{ | m{1cm} | m{13cm}| } 
\hline
\rowcolor{lightgray} \textbf{Code} & \textbf{Definition} \\ 
\hline
 \textbf{REQ} & Requirement abbreviation.  \\ 
\hline
 \textbf{SE} & Requirement Domain Abbreviation Code correspond with each requirements. (see Table 2)\\   
\hline
 \textbf{\#} & Requirement number ID \\ 
\hline
 \textbf{X} & Development Stage Encoding (see Table 3)\\ 
\hline
\end{tabular}
\caption{Requirement Encoding}
\end{table}

\bgroup
\def\arraystretch{1.5}
\begin{table}[H]
\centering
\begin{tabular}{ | m{7cm} | m{7cm}| } 
\hline
\rowcolor{lightgray} \textbf{Requirement Domain} & \textbf{Abbreviation Code} \\ 
\hline
 General & GE\\ 
\hline
 Hardware & HW\\ 
\hline
 Electrical & EC\\  
\hline
 Software & SW\\ 
\hline
 Performance & PE\\ 
\hline
 Safety & SF\\ 
\hline
 Sustainability & SU\\ 
\hline
\end{tabular}
\caption{Requirement Domain Abbreviation Code}
\end{table}

\bgroup
\def\arraystretch{1.5}
\begin{table}[H]
\centering
\begin{tabular}{ | m{7cm} | m{7cm}| }
\hline
\rowcolor{lightgray} \textbf{Development Stage} & \textbf{Encoding} \\
\hline
Proof of Concept & C\\
\hline
Prototype & P\\
\hline
Final Product & F\\
\hline
\end{tabular}
\caption{Development Stage Encoding}
\end{table}	














