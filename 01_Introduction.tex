%\documentclass[11pt]{article}
%\usepackage[document]{ragged2e}
%\usepackage[a4paper, total={6in, 9in}]{geometry}
%\usepackage{graphicx}
%\usepackage{float}
%\usepackage{multirow}
%\usepackage{array}
%\begin{document}




\setcounter{section}{0}
\section{Introduction}
\bigskip
\subsection{Background}
\break



\subsection{Scope}
This requirement specification document outlines functional and non-functional requirements expected of the Akrivia Beacon product through three different phases of development as shown in Figure 1. The three different phases  including: the proof-of-concept phase, prototype phase, and final product phase.
\medskip

\begin{figure}[H]
\centering
    \includegraphics[scale=0.4]{./images/dev-path.png}
    \caption{Development Cycle}
\end{figure}

A high level design of the system hardware and software is also presented to demonstrate the overall system architect and functionality of the product. The requirement section of this document is divided into five main categories as shown in the numbered list below. These requirements will indicate the constraints, demands, necessities, needs, and parameters that must be met or satisfied within the project time frame.

\begin{enumerate}
	\item General requirement 
	\item Hardware requirement 
	\item Electrical requirement 
	\item Software requirement 
	\item Performance requirement 
\end{enumerate}
\medskip

Furthermore, any relevant engineering standards, safety and sustainability practices are illustrated to demonstrate how the engineers at TRIWAVE will ensure that the final product is both economically and environmentally safe and sustainable. By following strict engineering standards and proper engineering practices TRIWAVE can guarantee that the final product will use materials, processes and services fit for their intended purposes.
\medskip
\medskip
\subsection{Intended Audience}
This document is presented by engineers at TRIWAVE as a guide for the functional and nonfunctional requirements of the Akrivia Beacon product. The intended audience of this document includes but not limited to, potential clients and/or partners, the supervising professors Dr. Craig Scratchley and Dr. Andrew Rawicz, associated teaching assistants and fellow TRIWAVE members. The hardware and software engineers of the project can reference this document during the development and/or testing stages of the project for clarification.
\break



\subsection{Requirement Classification}
\bigskip
For consistency purposes, the following requirement classification code convention is used to describe and organize requirements listed throughout this document. 
\begin{center}
\bigskip
	\textbf{[ REQ. SE.SSE.\# - X ]} 
\end{center}

\bgroup
\def\arraystretch{1.5}
\begin{table}[H]
\centering
\begin{tabular}{ | m{1cm} | m{13cm}| } 
\hline
\textbf{Code} & \textbf{Definition} \\ 
\hline
 \textbf{REQ} & Requirement abbreviation.  \\ 
\hline
 \textbf{SE} & Requirement Domain Abbreviation Code correspond with hierarchical of the requirements. (see Table 2)\\ 
\hline
 \textbf{SSE} & Requirement Domain Abbreviation Code for subsection of the requirements. (see Table 2) \\  
\hline
 \textbf{\#} & Requirement number ID \\ 
\hline
 \textbf{X} & Development Stage Encoding (see Table 3)\\ 
\hline
\end{tabular}
\caption{Requirement Encoding}
\end{table}

\bgroup
\def\arraystretch{1.5}
\begin{table}[H]
\centering
\begin{tabular}{ | m{7cm} | m{7cm}| } 
\hline
\textbf{Requirement Domain} & \textbf{Abbreviation Code} \\ 
\hline
General & GE\\ 
\hline
Hardware & HW\\ 
\hline
 Electrical & EC\\  
\hline
 Software & SW\\ 
\hline
 Software - UI & UI\\ 
\hline
 Performance & PE\\ 
\hline
Performance - Signal & SG\\ 
\hline
\end{tabular}
\caption{Requirement Domain Abbreviation Code}
\end{table}

\bgroup
\def\arraystretch{1.5}
\begin{table}[H]
\centering
\begin{tabular}{ | m{7cm} | m{7cm}| } 
\hline
\textbf{Development Stage} & \textbf{Encoding} \\ 
\hline
Proof of Concept & C\\ 
\hline
Prototype & P\\ 
\hline
Final Product & F\\  
\hline
\end{tabular}
\caption{Development Stage Encoding}
\end{table}	
	
	
	
%\end{document}



















