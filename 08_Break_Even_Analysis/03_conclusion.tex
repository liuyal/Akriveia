

\setcounter{section}{2}
\section{Conclusion}

\bigskip
In order for the Akriveia Beacon system to become a successful and economically sustainable product, a break-even analysis was performed to categorising production costs between those which are variable and those that are fixed. The break-even point is key in determining TRIWAVE SYSTEMS’ financial plan; by calculating the fixed and variable costs for the product, the minimum product units needed to be sold at a given price point to break-even can be obtained. This is an integral part of budgeting process which includes any business and operational cost and will be important for the company to continue development and production. From the break-even analysis the break-even point was determined for the gamma prototype to be 2 units and 62 units for mass production. 

\bigskip
The Gamma prototype had a very high variable cost with \$127.55 for Beacons and \$74.56 for ID Tag, since most components were purchased individually from distributors. Furthermore, the prototype had a very low fixed cost since there are no major cost associated with manufacturing, fabrication, or assembly which contributed to a low break-even point. As all hardware components were assembled by the team using available equipment supplied by SFU. However, considering the time and effort allocated for assembly of the gamma prototype which was around two days for three Beacons and two ID Tags, and the inconsistent quality of the final work, it is unreasonable to mass produce using these production methods.

\bigskip
For mass production, the variable cost decreased dramatically to \$36.70 for Beacons and \$22.78 for Tags. Since the components are ordered in bulk directly from the manufacturer and assembled on custom integrated PCB. However, fixed cost associated with mass production increased greatly taking in considerations for manufacturing, fabrication, assembly, and material cost. But through mass production the assembly process can be streamlined to produce a higher quantity product in much less time with better quality as well. Furthermore, by lowering sale prices higher sale numbers can be achieved, therefore, allowing the company to reach break-even point sooner.

\bigskip
However, it should also be noted that the break-even analysis should not act as a predictor of demand.
If the current market does not have high demand or the product goes into the market with the wrong price or wrong target audience, it will be very difficult to ever hit the break-even point. Which is why it is also critical to understand the current markets and to develop a strong Market Entry Strategies. This will allow for TRIWAVE SYSTEMS to become a successful company and is an important factor for investment recognition. As a product that could potentially change the outcome of disaster relief operations for thousands of people, the Akriveia Beacon is designed with the utmost care. As aforementioned, TRIWAVE SYSTEMS is dedicated to create a reliable and robust system design to improve disaster search and rescue operations with human safety as the pivotal focus.

