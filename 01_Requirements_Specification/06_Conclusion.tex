

\setcounter{section}{5}
\section{Conclusion}
\bigskip
As a company TRIWAVE SYSTEMS is focused on creating the most reliable and accurate indoor location rescue system. As aforementioned, Akriveia Beacon is a system of Ultra-WideBand (UWB) radio beacons and ID tag systems communicating via UWB and using trilateration to accurately obtain near real-time location of personnel within buildings during the event of a disaster. The location information which then can be reported to emergency responders and operators to provide accurate and reliable information for the search and rescue effort. 

\bigskip
The system overview, design, and constraints of the Akriveia Beacon were clearly established and a detailed outline of the requirements specifications was provided. Functional and non-functional requirements expected of the Akriveia Beacon product through three different phases of development are outlined, including: the proof-of-concept (completed August 2019), prototype, and final product (completed December 2019).

\begin{enumerate}
\setlength\itemsep{0.25em}
	\item General requirement 
	\item Hardware requirement 
	\item Electrical requirement 
	\item Software requirement 
	\item Performance Requirements
	\item Safety requirement 
	\item Sustainability  requirement 
\end{enumerate}


Since the Akriveia Beacon product is aimed to operate in emergency disaster scenarios, various engineering standards and safety requirements must be followed to ensure usability, durability and acceptability in the market. By following the Cradle to Cradle development cycle, the product is ensured to be both innovating and environmentally sustainable. Lastly, to ensure optimal product quality control a brief Alpha stage test plan is included in the appendix at the end of this document. It details the testing procedure for the proof of concept prototype on various parts of the Akriveia Beacon system.

\bigskip
While going through the three phases of development, this document will provide a reliable
reference to ensure requirements are satisfied at each milestone, as well as to provide a strict criteria in which to compare with the final product.

%\end{document}