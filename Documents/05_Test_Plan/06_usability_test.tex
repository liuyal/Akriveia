

\setcounter{section}{5}
\section{Usability Test Plan}
\bigskip
The Akriveia Beacon’s web server user interface will undergo usability testing to determine the state of its usability at various stages of development. The usability testing outlines testing procedure that will be done by the engineering and design team using heuristic usability evaluations. Each evaluator will independently examine the UI and check for compliance and usability. After collecting the results the team will discuss and compile possible solutions to usability issues and generate a list of solutions. Finally, the redesign will be implemented and regression testing will be done. Analytical usability testing will take place during the prototype and final product phases of development as the user interface is well defined during these two stages. The testing procedure will follow the steps described below.

\bigskip

\textbf{Step 1:} Usability Research Data Collection\\
\medskip
The first step is to collect data generated by the usability test. Each evaluator will perform tasks outlined in the analytical usability testing procedure. From the procedures performed issues will be highlighted and documented. Each issue will have the following:
\begin{itemize}
\setlength\itemsep{0.1mm}
	\item An issue identification (ID).
	\item Note where it happened (screen, module, UI widget, flow, etc.).
	\item Task the user was engaging in.
	\item Concise description of the issue.
\end{itemize}
Data collected will be shown in a table similar to the table below:

\def\arraystretch{1.5}
\begin{table}[H]
\centering
\begin{tabular}{ | p{0.5cm} | p{2cm}| p{5cm} | p{5cm} | p{0.5cm} | p{0.5cm} | p{0.5cm}|} 
\hline

\rowcolor{lightgray} \multicolumn{1}{|l|}{\textbf{ID}} & \textbf{Where} & \textbf{Task} &  \textbf{Description} & \textbf{P1} & \textbf{P2} & \textbf{P3} \\ 
\hline
1 & Login Page & Login with wrong Password & No error message for wrong user name input & X & - & - \\
\hline
2 & Map View & Click on beacon icon & Beacon info text too small & - & X & X \\
\hline
\end{tabular}
\caption{Usability Test Results}
\end{table}	



\textbf{Step 2:} Issue prioritization\\
\medskip
Once sufficient testing has been performed by evaluators of the team, issues must be prioritized as time and resources are limited for this project. Each usability issue receives a grade of severity, influenced by factors such as:
\begin{itemize}
\setlength\itemsep{0.1mm}
	\item Task criticality: Impact on user if the task is not accomplished.
	\item Issue frequency: How many times an issue has occurred with various participants.
	\item Issue impact: How much has it impacted the user trying to accomplish the task.
\end{itemize}

\bigskip



\textbf{Step 3:} Solution Generation\\
\medskip
After usability testing is performed following test cases described in section 5, using the combined feedback and evaluations, the engineers at TRIWAVE SYSTEMS will re-evaluate possible UI designs for each usability issue that occurred during testing to determine the best and optional solution. A list of recommendations and solutions will be generated with usability test results. For each design decision several alternative solutions must be generated to include other possible ways to address the issue. 

\pagebreak

Once internal usability testing is complete external usability testing will be carried out in cycles with real users consists of volunteer participants ideally relating to targeted audience. The first test cycle occurs near the end of the prototype phase and the second test cycle occurs near the end of the Final Product phase. Testing will be done with two small groups of participants that are unfamiliar with the project development environment. First group will be asked to perform usability test cases outlined in section 5. An observer will document actions and observations of the testing process as well as to keep note of average time to complete each task, the amount of errors and error rate, number of tasks completed, and perform a sequence analysis. Issues will be represented similar to the method mentioned previously. With the collected data the designers will re-evaluate the user interface and develop further design solutions for potential issues. After re-design and implementation of a second small group of participants will be asked to perform the same tasks as the first group to finalize design and changes.

\bigskip
From the results generated by participants the following usability elements will be addressed throughout the two testing cycles and development stages. 

\medskip
\begin{itemize}
\setlength\itemsep{0.1mm}
	\item \textbf{Easability:} The familiarity and intuitiveness of the system and how comfortable the users are with the user interfaces in general. 
	\item \textbf{Navigation:} The reliability of the navigation sequences are, how easy is it for the users to understand paths, and/or short cuts. Can the users easily retrace their steps or go back to previous states if they have made a mistake?
	\item \textbf{Responsiveness:} Does the users receive sufficient feedback from interacting with the system? 
	\item \textbf{Intuitiveness:} How quickly can a new user familiarize themselves with the user interface? Whether or not the users are able to perform tasks within a certain amount of time?
	\item \textbf{Robustness:} Safety and reliability of the device and system are addressed by eliminating or minimizing potential error (slips and mistakes) and enabling error recovery.
\end{itemize}

\medskip
By following these usability testing procedures mentioned above, the engineers and designers at TRIWAVES SYSTEMS can ensure a reliable and intuitive user interface will be produced to meet the needs of its end users. 