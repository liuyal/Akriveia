

\subsection{Introduction}
\medskip
The User Interface Design Appendix provides a detailed description and analysis of the Akriveia Beacon System in terms of the design, communication, and operations with the intended users. The Akriveia Beacon system is an advanced indoor location rescue system designed to aid first responders during an emergency search and rescue situation by providing accurate location of trapped victims within commercial buildings. The primary users to interact with the Akrievia Beacon system will be first responders such as fire fighters or emergency management personnel. The secondary users to interact with the system would be administrators or IT technicians that would utilize or perform maintenance and upkeep of the system. Lastly, the tertiary users would be the employees of the company using the Akriveia Beacon system. The employee will only interact with the system by turning the ID tags on during an emergency. Since the system is intended to operate under extremely stressful environments and situations, the user interface must be as clear and intuitive to use as possible to ensure that first responders can operate at peak efficiency along side the Akriveia Beacon system.
\bigskip

\subsubsection{Purpose}
\medskip
The focus of this user interface design appendix is to act as a reference for engineers at TRIWAVE SYSTEMS throughout development. In order to create an interface that is both clear and intuitive for the intended user, the hardware and software interfaces must follow strict design standards and requirements. Such standards and requirements will ensure that during the intended operating scenario, the system would not cause users unexpected error due to implications of insufficient operating knowledge or unforeseeable circumstances. These design requirements will be presented in conjunction with the three specific development phases: the proof-of-concept phase, prototype phase, and the Final product phase.
\bigskip

\subsubsection{Scope}
\medskip
This document section includes detailed overview of user and technical analysis, engineering safety and standards, and usability testing in order to provide sufficient understanding of the user interface for the Akriveia Beacon system. As a system to be operating under disaster or emergency situations, the Akriveia Beacon must require some form of basic user knowledge in order for different parties of the user base to operate the system sufficiently. Outline of the required user knowledge allowing basic usage of the Akriveia Beacon system will be presented. As well as the following seven fundamental technical analysis principles will be considered when making design choices for the user interface: discoverability, feedback, conceptual models, affordances, signifiers, mappings, and constraints; as outlined from Don Norman's The Design of Everyday Things \cite{R10-3-1}. Finally, the appendix will include analytical and empirical system test plans with different scenarios that is aimed at testing how the Akriveia Beacon system would operate under each specified condition during each stage of the development cycle. The test cases covered will provide additional quality assurance of the final product to ensure that the Akriveia Beacon system is both accurate and reliable for its intended purpose.






