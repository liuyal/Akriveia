

\subsection{Technical Analysis}
\bigskip
This section will analyze the consideration for Seven Elements of UI Interface as outlined by Don Norman for the Akriveia Beacon system; which includes the following design factors, discoverability, feedback, conceptual models, affordances, signifiers, mappings, and constraints \cite{R10-3-1}. By incorporating these design element in to the system, the usability and quality of the final product can be substantially improved. 
\medskip

\subsubsection{Discoverability}
\medskip
Discoverability: Is it possible to even figure out what actions are possible and where and how to perform them?  In the context of product and interface design, discoverability is the degree of ease with which the user can find all the elements and features of a new system when they first encounter it \cite{R10-3-2}. The Akrivia beacon system is designed to operate under emergency situations and disasters, which means that it is paramount that the UI creates discoverability for its users. The overall interaction should be simple enough for each level of users to comprehend and understand without the need for much interpretation. 

\bigskip
Primary user - First Responders’ main point of interaction with the Akriveia Beacon system is through the GUI. The GUI will be simple with intuitive design allowing for quick understanding of the system and any relating concepts. For prototype UI design, the interface will be quick to access  and contains a scaled blueprint of the structure with colored indicators will conveying the location of victims on the map view (similar to figure \ref{map_ff}). 

\bigskip
Secondary user - system and IT administrators will have more in depth access to the system. Since they are required to manage profiles related to each employee and their associated ID tag, the GUI needs to be simple and robust (see section 10.4.2 UI Mock-Ups). Actions such as adding, removing, and editing profiles or system configuration must be intuitive. 

\bigskip
Tertiary users - The ID tags are small in design and resembles an access card so the user know how to wear the device. In the case of an emergency the employees must trigger a simple touch button to enable broadcasting of current location to the beacons. Otherwise the user must remember to keep ID tag on person while on company property.
\medskip


\subsubsection{Feedback}
\medskip
Feedback - There is full and continuous information about the results of actions and the current state of the product or service. After an action has been executed, it is easy to determine the new state. As the main layer of interaction between the users and the system, the GUI must provide visual indicators for any actions performed. Indicators such as confirmation messages and UI element state changes will be shown. Most importantly, during an emergency the system will be in active mode, icons and indicators on the GUI must be updated in near real time to provide constant visual feedback to the users. Furthermore, beacons and ID tags will also provide visual feedback to its users via simple LEDs indicate active or inactive system; as well as other information such as battery level or state of data transmission. 
\pagebreak


\subsubsection{Conceptual models}
Conceptual Models - The design projects all the information needed to create a good conceptual model of the system, leading to understanding and a feeling of control. The conceptual model enhances both discoverability and evaluation of results. The Akriveia Beacon system for primary users creates a conceptual model in the form of a floor plan represented through the GUI. Primary users will easily be able to relate the model to the real life layout of the building and operate accordingly. For secondary users, the system model function much like any web application that they have previously encountered allow them to easily navigate the UI and perform necessary tasks. For tertiary users the ID tag device only has one button and a few status indicating LEDs for clear interactions.

\medskip
\subsubsection{Affordances}
Affordances - The proper affordances exist to make the desired actions possible. Clarity of the design creates a relationship between the look and intended use of the product which allows users to quickly understand the correct operations for the system. Some affordances of Akriveia Beacon are: Beacons and ID tags have clear LED indicators to display system status; The ID tags have bright colored button to indicate the interaction point for the user; The GUI have clear and concise labels, text, and color to simplify interactions with the user; The GUI will show location of ID tags clearly with colored indicators to identify location.

\medskip
\subsubsection{Signifiers}
Signifiers - Effective use of signifiers ensures discoverability and that the feedback is well communicated and intelligible. Basic colors such as green, amber, and red are used to indicate system status such as good, okay, and bad. These three basic colors will be used throughout the product to indicate system status. LEDs on the Beacon and ID tags will use indication colors to show device status. The GUI will use initiation colors to show system status and ID tag last ping time.

\medskip
\subsubsection{Mappings}
Mappings - The relationship between controls and their actions follows the principles of good mapping, enhanced as much as possible through spatial layout and temporal Contiguity. Some examples on the system include: activated tabs on the GUI are underlined with a bold color to show users the current selected tab. Pop up messages and text box will be provided when users interact with the GUI to show the UI element that they are interacting with. The ID tags and Beacons will output device status using LEDs whenever a system change occurs due to user interaction. This allows the user to receive feedback from the devices.

\medskip
\subsubsection{Constraint}
Constraints - Providing physical, logical, semantic, and cultural constraints guides actions and eases interpretation. Adding constraints to the design will limit the number of actions that the user can perform with the device. One major constraint is that once the Akriveia Beacon system is activated it can not be deactivated until the situation is resolved. Only a high level system administrator will be able to access and disable the system. This constraint prevents having the system shutdown accidentally or unexpectedly. Another constraint is ensuring the system GUI is accessible on a tablet device. Mobile devices actions have to be limited inorder to provide an intuitive user interaction since the only input are through a touchscreen. This constraint creates intuitive interaction between the GUI and the users.
