

\setcounter{section}{0}
\section{Introduction}

\bigskip
The Akriveia Beacon by TRIWAVE SYSTEMS focuses on improving the localization and rescue
process of personnel trapped in buildings during or after small scale disasters. These disasters may range from
small fires to low magnitude earthquakes. This is achieved by incorporating a combination of advanced Ultra-
wide-band radio modules, micro-controller units and data processing units to create a dependable
indoor positioning system using trilateration techniques. Such trilateration techniques have been reliability 
used in the search and rescue industry for decades. 

\bigskip
There are three main development phases for the Akriveia Beacon system including: proof-of-concept phase, prototype phase, and final product phase. Throughout each phase of development various design changes are made to improve the functionality and reliability of the system. Currently in the refined prototype phase there are major component addition and changes which modifies the core functionalities of the beacon components. One major optimization of the beacon system is that the transceiver for ranging switched from  Bluetooth low energy to ultra-wideband. Other additions include WiFi modules for networking capabilities, backup battery packs to mitigate external power failures, and beacon casing designed to provide security and structural integrity. 

\bigskip
The following document discusses a brief outline of all necessary design changes and optimization for the refined prototype and production-ready version of the Akriveia Beacon system, as well as to provide relevant standards applied to the proposed designs. Furthermore, an assessment of potential risks and any safety issue is presented and user interaction and usability of design are addressed.

