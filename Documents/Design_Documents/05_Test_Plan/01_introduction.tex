

\setcounter{section}{0}
\section{Introduction}
\bigskip

The following document contains the detailed Test Plan for Akriveia Beacon - The Indoor Location Rescue System created by TRIWAVE SYSTEMS. The Akriveia Beacon system focuses on improving the location and rescue process of personnel trapped in buildings during and after small scale disasters such as fires and low magnitude earthquakes. 

\bigskip
Such a system allows for search and rescue operations to be carried out more safely and reliably. The core functionality allows for trapped victims to be located  in a disaster situation confined within complex urban environments.  By pinpointing the exact location of any victim wearing an ID tag, the search and rescue time for first responders is minimized; which is crucial in any disaster rescue operations. This Is achievable by incorporating combinations of advanced Ultra-wide-band radio modules and microcontrollers to create a dependable indoor positioning system using trilateration methods. 

\bigskip
This test plan is divided into four main sections: system, hardware, software, and electrical tests. Each test plan section will detail the testing procedure and provide comprehensive test cases derived from design requirements that directly validate primary and secondary functionality of each of the major components and the overall system of Akriveia Beacon. Furthermore, this document will outline major functionalities and features and show the quantifiable, observable, and realistic factors that are incorporated into the design of the system. All test cases for each of the four test plans follows the below format.

\medskip
\begin{center}
	\textbf{[ TEST-\#]} 
\end{center}


\bgroup
\def\arraystretch{1.3}
\begin{table}[H]
\centering
\begin{tabular}{ | m{1cm} | m{13cm}| } 
\hline
\rowcolor{lightgray} \textbf{Code} & \textbf{Definition} \\ 
\hline
 \textbf{TEST} & Test Plan abbreviation. \\ 
\hline
 \textbf{\#} & Test Case ID \\ 
\hline
\end{tabular}
\caption{Test Case Encoding}
\end{table}

Each test case is grouped into four sections shown below and their IDs correspond with a letter representing the system component. 


\bgroup
\def\arraystretch{1.3}
\begin{table}[H]
\centering
\begin{tabular}{ | m{7cm} | m{7cm}| } 
\hline
\rowcolor{lightgray} \textbf{Component} & \textbf{Abbreviation Code} \\ 
\hline
 System Level & SL\\ 
\hline
 Hardware & HW\\ 
\hline
 Software & SW\\  
\hline
 Electrical & EC\\
\hline
 Usability & US\\ 
\hline
\end{tabular}
\caption{Planning Stage Abbreviation Code}
\end{table}


\bigskip
Testing equipment and facility needed are provided by SFU and are available for the test team to use and perform. TRIWAVE SYSTEMS is dedicated to create a reliable and robust indoor tracking system, designed to improve disaster search and rescue operations with human safety as the pivotal focus.

