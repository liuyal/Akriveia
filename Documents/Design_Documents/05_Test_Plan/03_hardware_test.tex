

\setcounter{section}{2}
\section{Hardware Test Plan}
\bigskip

The system consists of two major hardware components, the anchor beacons, and the wearable ID tags. Each beacon contains a DWM1000 Ultra-wideband (UWB) transceiver module, a 3.3V 8MHz Arduino Pro Mini (PM33), and an ESP32 WiFi module, and each ID tag contains a DWM1000 module and an Arduino Pro Mini. Hardware components must be tested individually to verify their primary functions and performance.


\bigskip
\bgroup
\def\arraystretch{1.3}
\begin{table}[h!]
    \centering
    \begin{tabular}{|p{0.07\linewidth}|p{0.45\linewidth}|p{0.40\linewidth}|}  
    \hline
    ID & Test Procedure & Validation\\ 

    \hline
    HW-01
	& DWM1000 UWB to PM33 connections are valid with Serial Peripheral Interface (SPI). Connect Beacon to secondary PC via serial connection.
	& Reset PM33 by pushing toggling the reset button on beacon. Serial connection should show device initiation message with correct device ID: Device ID: DECA\\
     
    \hline  
    HW-02
	& Configure 1 beacon and 1 ID tag in ranging mode. Set 2 device 1m apart measured by tape measure. Start ranging and observe results using serial connection from beacon.
	& Ranging values can be observed from serial connection on beacon device. The values are within 0.5m tolerance.\\

	\hline  
	HW-03
	& Multi-Tag detection ability. Configure 1 beacon and at least 2 ID tags in ranging mode. Set 1 id tag 1m apart from beacon and another 2m apart from beacon measured by tape measure. Start ranging and observe results using serial connection from beacon.
	& Ranging values can be observed from serial connection on beacon device. The values are within 0.5m tolerance. Ranging messages show 1 id tag with 1m range values and the other id tag with 2m range values. Unique tag id is also matched correctly with each range. \\
	
	\hline  
	HW-04
	& Beacon can be started by sending start command via serial. Configure 1 beacon and 1 ID tag in ranging mode.
	& Connect to beacon using serial connection.
	On PC connected to beacon via serial, send command start, the beacon returns \texttt{start\_ack} to acknowledge command and begins receiving range values.\\
	
	\hline  
	HW-05
	& Beacon can be stopped by sending end command via serial. Configure 1 beacon and 1 ID tag in ranging mode.
	& Connect to beacon using serial connection.
	On PC connected to beacon via serial, send command end, the beacon returns \texttt{end\_ack} to acknowledge command and stops receiving range values from ID tag.\\
	
	\hline  
	HW-06
	& Beacon on/off state is saved in non-volatile memory. Configure 1 beacon and 1 tag in ranging mode. Start ranging with command start. Power cycle beacon.
	& Before power cycle beacon outputs correct ranging values, after power cycle beacon initialization messages are shown and continues to output ranging values.\\

    \hline
    \end{tabular}
    \caption{Hardware Test Cases - Part 1}
\end{table}

\pagebreak
\
\bgroup
\def\arraystretch{1.3}
\begin{table}[h!]
    \centering
    \begin{tabular}{|p{0.075\linewidth}|p{0.45\linewidth}|p{0.45\linewidth}|} 
    \hline
    ID & Test Procedure & Validation\\ 

    \hline  
    HW-07
	& ESP32 is configured to connect to WiFi access point and receives IP with pre-specified port. Secondary PC device is also connected to the same network and receives IP. IP on PC is host IP.
	& Using netcat commands on secondary PC device to send and listen for UDP packets.
	All packets are sent to ESP32 properly and can be observed via serial connection from ESP32.\newline
	Listen: nc -ulk 0 \textless PORT \textgreater \newline
	Send: nc -u \textless IP \textgreater \textless PORT \textgreater \\
     
    \hline  
    HW-08
	& Hardware serial between PM33 and ESP32 is established by connecting: \newline
	ESP32 G17 -\textgreater PM33 RXD \newline
	ESP32 G16 -\textgreater PM33 TXD \newline
	ESP32 GRD -\textgreater PM33 GRD \newline
	ESP32 is connected to secondary PC using USB cables.
	& Hardware serial communication between PM33 and ESP32 verified by resetting PM33 and observing PM33 reboot message on ESP32 side. Using COM port viewer the reboot message PM33 can be seen from ESP32 serial out on secondary PC\\

	\hline  
	HW-09
	& Beacon can be pinged by server. Beacon is connected to WiFi access point and web server is configured to connect to the same access point and send ping command.
	& Server issues ping, beacons receives and sends packet to host IP. Beacon returns packet containing ping acknowledgement and basic device information.
	Packet should be \texttt{ping\_ack} , \textless EUI \textgreater\\
	
	\hline
	HW-10
	& Beacon can be rebooted by server. Beacon is connected to WiFi access point and web server is configured to connect to the same access point and send reboot command.
	& Server issues reboot, beacons receives and sends 2 reboot acknowledgement  packets back to server. Packets are:
	\texttt{esp\_reboot\_ack}  \newline
	\texttt{pm33\_reboot\_ack}  \\
	
	\hline 
	HW-11
	& Beacon ranging delay is less than 5 seconds. Configure 1 beacon and 1 ID tag in ranging mode. Set 2 device 1m apart measured by tape measure. Start ranging and observe results using serial connection from beacon.
	& Observe ranging values from beacon via serial connection with time stamp option enabled. Time difference between each ranging value must be less than 5 seconds. \\
	
	\hline
	HW-12
	& Beacon ranging delay is less than 5 seconds. Configure 3 beacons and 1 ID tag in ranging mode. Set each beacon-tag pair 1m apart measured by tape measure. Start ranging and observe results using individual serial connections from each beacon.
	& Observe ranging values from beacon via serial connection with time stamp option enabled. Time difference between each ranging value must be less than 15 seconds. \\
	
	  \hline
    \end{tabular}
    \caption{Hardware Test Cases - Part 2}
\end{table}

\begin{table}[h!]
    \centering
    \begin{tabular}{|p{0.075\linewidth}|p{0.45\linewidth}|p{0.45\linewidth}|} 
    \hline
    ID & Test Procedure & Validation\\ 
	
	\hline
 	HW-13
    	& Observe the time for the ESP32 to successfully connect to the configured SSID broadcasted by the dedicated WiFi access point and assigned an IP with configured port. Secondary PC device connected to the ESP32 is used to monitor the setup messages output on COM port viewer.
	& With time stamp option enabled in the COM port viewer, record the time between the appearance of serial messages "Connecting to WiFi" to "UDP port, \textless PORT \textgreater". Expected time is less than 4 seconds. \\
	
	\hline
	HW-14
	& Default DW1000 PRF config is 16MHz. Analyze ranging improvements by increasing pulse repetition frequencies from 16MHz to 64MHz. Replace \texttt{FREQ\_16MHZ} to \texttt{FREQ\_64MHZ} in the PulseFrequency field	
	& Set a beacon and a tag 1m apart measured by tape measure. Observe ranging values from beacon via COM port viewer with PRF at 16MHz, then observe ranging values from beacon via COM port viewer with PRF at 64MHz. Compare their ranging values, the latter setup should yield higher accuracy by 0.1m \\
	
	\hline
	HW-15
	& Observe the time it takes for the PM33 reset and the ESP32 to restart upon issue of the reboot command through serial.
	& Configure ESP32 to connect to WiFi access point and receive an IP with pre-specified port. Connect a secondary PC to the same SSID and assign it the host IP. Using netcat commands on shown in HW-07, issue the reboot command in ranging mode and observe the \texttt{pm33\_reboot\_ack} and \texttt{esp\_reboot\_ack} appear in chronological order. Record the time taken from the issue of the reboot command to when a ranging value first appears again. \\
	
	\hline
	HW-16
	& Observe that the DW1000 device configuration chooses one of the recommended preamble codes as a result of increasing pulse frequency to 64MHz from HW-14 and selecting UWB Channel 5. Based on user manual, recommended preamble codes for 64MHz PRF on Channel 5 are 9, 10, 11, 12.
	& View serial message output from COM port viewer. In the Device mode line of the device initiation message, PRF should be 64MHz, Channel should be \#5 and Preamble code should be 9, 10 ,11 or 12. \\
	
	\hline
	HW-17
	& Calibrate antennas between beacon-tag pairs by finding optimal AntennaDelay value. To find optimal AntennaDelay value, place a beacon-tag pair 5.01m apart, given that both DW1000 are configured with 64MHz PRF and UWB Channel 5. Beacon acts as the initiator while tag acts as the responder.
	& Monitor the COM port viewer of the responder, record the AntennaDelay value when its associated range value is within the predefined epsilon value of 5.01m. \\
	
    \hline
    \end{tabular}
    \caption{Hardware Test Cases - Part 3}
\end{table}

	
	
	

