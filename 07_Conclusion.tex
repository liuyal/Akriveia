

\setcounter{section}{6}
\section{Conclusion}

\bigskip
As urban centers around the world experience rapid growth and changes so does the risk of being potentially getting trapped within buildings during disasters. The time period right after a disaster strikes is the most critical time for saving victims lives. In current practices, first responders have limited time to evaluate the situations when they arrive on scene of disaster and must take crucial actions accordingly. Searching the incident building for possible victims is one of the major tasks undertaken by first responders after an incident occurs. The lack of timely information could be the difference between life and death in such situations.

\medskip
As such, an reliable and accurate indoor location rescue system is needed to aid first responders locate trapped personal. Akriveia Beacon is a system of anchor beacons and ID tags controlled by microcontrollers and communicating via UWB, along with trilateration algorithm to accurately obtain near real time location of trapped personnels within buildings during the event of a disaster. The location data is then reported to a portable data processing unit using a closed Wi-Fi network, which can be interact with directly by emergency first responders and operators to provide accurate and reliable information for the search and rescue effort. 

\medskip
This document outlines a detailed design specifications and it is intended to be used as a design
reference for the engineers at TRIWAVE SYSTEMS, as well as to provide detailed insight for the hardware, software designs required for the Akriveia Beacon product. The system overview, design, and constraints of the Akriveia Beacon system are clearly established; as well as to present a detailed outline of the system design specifications. These design specification outlines high level system architectures, system functions and implementation of the Akriveia Beacon product through three different phases of development including: the proof-of-concept (completed August 2019), prototype, and final product (completed December 2019).


