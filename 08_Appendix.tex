%\documentclass[11pt]{article}
%\usepackage[a4paper, total={6.5in, 9.5in}]{geometry}
%\usepackage[document]{ragged2e}
%\usepackage{lipsum}
%\usepackage{graphicx}
%\usepackage[noadjust]{cite}  
%\usepackage{float}
%\usepackage[numbib]{tocbibind}
%\usepackage{multirow}
%\usepackage{array}
%\usepackage{setspace}
%\usepackage{cellspace}
%\usepackage{etoolbox}
%\usepackage{scrpage2}
%\usepackage{longtable}
%\usepackage[table, svgnames]{xcolor} 
%\usepackage{titlesec}
%\usepackage{amsmath}
%\setcounter{secnumdepth}{4}
%\titleformat{\paragraph}
%{\normalfont\normalsize\bfseries}{\theparagraph}{1em}{}
%\titlespacing*{\paragraph}
%{0pt}{3.25ex plus 1ex minus .2ex}{1.5ex plus .2ex}
%\ifoot[]{}
%\cfoot[]{}
%\ofoot[\pagemark]{\pagemark}
%\pagestyle{scrplain}
%\begin{document}

\setcounter{section}{7}
\section{Appendix}

\subsection{Proof Of Concept(PoC) Acceptance Test Plan}

% What testing is there supposed to be?
The PoC Acceptance Test Plan includes all the testing procedures for verifying and validating the requirement specifications under a formal test environment. The goal of the PoC Acceptance Test Plan is to ensure the basic requirements stated in the \textbf{\textit{System Requirements}} section of the document are met.

\medskip
Three main goals for the PoC testing are as follows:

\begin{enumerate} 
    \item 2.4 GHz chips are able to receive and transmit data accurately
    \item Arduino micro-controllers(MCU) receives the transmission data 
    \item Raspberry Pi receives data serially from Arduino MCU
\end{enumerate}

%The acceptance test plan will consists of the test below:
%\begin{enumerate}
%    \item System(General) Tests
%    \item Hardware Tests
%    \item Electrical Tests
%    \item Software Tests
%    \item Software - UI Tests
%    \item Performance Tests
%    \item Performance - Signal Tests
%\end{enumerate}
% Acceptance Test Plan is Long

\subsubsection{Systems (General Testing)}

\begin{table}[h!]
    \centering
    
    \begin{tabular}{|m{0.15\linewidth}|m{0.02\linewidth}|m{0.3\linewidth}|m{0.45\linewidth}|} 
    \hline
    \multicolumn{4}{|l|}{System (General) Test}  \\ 
    \hline
    REQ.ID & \multicolumn{2}{l|}{Testing Criteria} & Expected Output \\ 
    \hline
    % Row of req ID, #, Testing Criteria, Expected Output
    REQ.GE.1-C                  
    & 1
    & Ensure all tests conducted are representative of conditions indoor instead of outdoor conditions or all-terrain conditions 
    & System testing and design are representative of conditions within a building. 
    System testing does not consider the all-weather conditions nor any unpredictable outdoor conditions.\\ 
    \hline

    % Row of req ID, #, Testing Criteria, Expected Output
    \multirow{2}{*}{REQ.GE.2-C} 
    & 1 
    & System is able to switch between two modes: idle and emergency modes       
    & Switching between the idle and emergency modes are to be robust and reliable. 
    There is significant differences in power consumption, signal communication and content between the two modes. \\ 
    \cline{2-4}
    & 2 
    & Idle mode and emergency modes should show different operations 
    & MCU serial output should show different data transfer when system is in the different modes \\ 
    \hline

    % Row of req ID, #, Testing Criteria, Expected Output   
    \multirow{2}{*}{REQ.GE.3-C} 
    & 1 
    & System should determine the location of the ID tag on one floor of a building     
    & Software algorithms determine the relative location of the ID tag to the beacons. 
    Clear distances are measured between the beacons themselves or to the ID tag.~\\ 
    \cline{2-4}
    & 2 
    & System to show relative location of the ID tag     
    & A relative location of the ID tag to the approximate floor space is calculated \\
    \hline
    \end{tabular}
    \caption{PoC System(General) Requirement Test Plans}
\end{table}

\break
\subsubsection{Hardware Test Plan}

\begin{table}[h!]
    \centering
    \begin{tabular}{|m{0.15\linewidth}|m{0.02\linewidth}|m{0.3\linewidth}|m{0.45\linewidth}|} 
    \hline
    \multicolumn{4}{|l|}{Hardware Tests}           \\ 
    % -----DONT NEED TO CHANGE THIS SECTION -----
    \hline
    REQ.ID & \multicolumn{2}{l|}{Testing Criteria} & Expected Output         \\ 
    \hline
    % ----DONT NEED  TO CHANGE THIS SECTION ----
    % Row of req ID, #, Testing Criteria, Expected Output
    REQ.HW.1-C                  
    & 1 
    & Server should use SBC to be the automated product control
    & SBC should take all received data and perform calculations   \\ 
    \hline
   
    % Row of req ID, #, Testing Criteria, Expected Output
    REQ.HW.2-C                  
    & 1 
    & Beacons to use the MCU to be the automated product control
    & MCU should control signal transmission through the 2.4 GHz chips and be fully operational 
    when used as the automated product control         \\ 
    \hline
    
    % Row of req ID, #, Testing Criteria, Expected Output
    \multirow{2}{*}{REQ.HW.3-C} 
    & 1 
    & ID tags to use the MCU to be the automated product control
    & MCU should control signal transmission through the 2.4 GHz chips and be fully operational 
    when used as the automated product control            \\ 
    \cline{2-4}
    & 2 
    & MCU should receive signals from the beacons and transmit signal back through MCU  
    & Arduino serial output for ID tag should show data received and send proper data output        \\
    \hline
    
    % Row of req ID, #, Testing Criteria, Expected Output
    REQ.HW.4-C                  
    & 1 
    & SPI should be used as the interface with the MCU and transceiver
    & Arduino serial output for MCU should show proper data transfer through the 2.4 GHz transceiver   \\ 
    \hline
    
    % Row of req ID, #, Testing Criteria, Expected Output
    \multirow{2}{*}{REQ.HW.5-C} 
    & 1 
    & Beacons should use 2.4 GHz radio frequency as the receiver     
    & Correct data input received from the 2.4 GHz radio module          \\ 
    \cline{2-4}
    & 2 
    & Beacons should use 2.4 GHz radio frequency as the transmitter     
    & Correct data output transmitted from the 2.4 GHz radio module   \\
    \hline
    
    % Row of req ID, #, Testing Criteria, Expected Output
    \multirow{2}{*}{REQ.1HW.6-C} 
    & 1 
    & ID tags should use 2.4Ghz radio frequency as the receiver     
    & Correct data input received from the 2.4Ghz radio module          \\ 
    \cline{2-4}
    & 2 
    & ID tags should use 2.4 GHz radio frequency as the transmitter     
    & Correct data output transmitted from the 2.4 GHz radio module   \\
    \hline
    
    % Row of req ID, #, Testing Criteria, Expected Output
    \multirow{2}{*}{REQ.HW.7-C} 
    & 1 
    & Serial USB interface should be properly implemented for beacon to server communication
    & Beacon and server communicates and data transfer is functional         \\ 
    \cline{2-4}
    & 2 
    & Correct data should be transmitted through serial USB     
    & Correct data is transmitted or received through serial USB. Server or beacons should receive or transmit
    intended data   \\
    \hline    

\end{tabular}
    \caption{PoC Hardware Requirement Test Plans}
\end{table}



\break
\subsubsection{Electrical Test Plan}

\begin{table}[h!]
    \centering
    
    \begin{tabular}{|m{0.15\linewidth}|m{0.02\linewidth}|m{0.3\linewidth}|m{0.45\linewidth}|} 
    \hline
    \multicolumn{4}{|l|}{Electrical Tests}      \\ 
    % -----DONT NEED TO CHANGE THIS SECTION -----
    \hline
    REQ.ID   & \multicolumn{2}{l|}{Testing Criteria}     & Expected Output     \\ 
    \hline
    % ----DONT NEED  TO CHANGE THIS SECTION ----
    % Row of req ID, #, Testing Criteria, Expected Output
    REQ.EC.1-C                  
    & 1 
    & Beacons should be powered through 120 V AC, 60 Hz outlets
    & Beacons are powered by the AC outlets and is functional   \\ 
    \hline
    % Row of req ID, #, Testing Criteria, Expected Output
    REQ.EC.2-C                  
    & 1 
    & ID tags should have it's own 3.3 V battery power source
    & ID tags are battery powered and is functional  \\ 
    \hline
\end{tabular}
	\caption{PoC Electrical Requirement Test Plans}
\end{table}

\subsubsection{Software Test Plan}

\begin{table}[h!]
    \centering
    \begin{tabular}{|m{0.15\linewidth}|m{0.02\linewidth}|m{0.3\linewidth}|m{0.45\linewidth}|} 
    \hline
    \multicolumn{4}{|l|}{Software Tests }           \\ 
    % -----DONT NEED TO CHANGE THIS SECTION -----
    \hline
    REQ.ID      & \multicolumn{2}{l|}{Testing Criteria}      & Expected Output          \\ 
    \hline
    % ----DONT NEED  TO CHANGE THIS SECTION ----
    % Row of req ID, #, Testing Criteria, Expected Output
    \multirow{2}{*}{REQ.SC.1-C} & 1 
    & Server should receive ToF(Time of Flight) data from the beacons
    & Input of ToF data is received from the beacons        \\ 
    \cline{2-4}
    & 2 
    & ToF data received from beacons should be accurate  
    & ToF data is verified to be correct. Data is tested to be true and does not contain errors   \\
    \hline 
\end{tabular}
	\caption{PoC Software Requirement Test Plans}
\end{table}

%\subsubsection{Software - UI Test Plans}
%There are no Software - UI requirements in the PoC phase. Test plans will be developed for Prototype and 
%Final Product phases at a later date.

\subsubsection{Performance Test Plans}
\begin{table}[h!]
    \centering
    
    \begin{tabular}{|m{0.15\linewidth}|m{0.02\linewidth}|m{0.3\linewidth}|m{0.45\linewidth}|} 
    \hline
    % Change this line to match #. (Section) Test
    \multicolumn{4}{|l|}{Performance Tests    }      \\ 
    % -----DONT NEED TO CHANGE THIS SECTION -----
    \hline
    REQ.ID      & \multicolumn{2}{l|}{Testing Criteria}      & Expected Output          \\ 
    \hline
    % ----DONT NEED  TO CHANGE THIS SECTION ----
    % Row of req ID, #, Testing Criteria, Expected Output
    \multirow{2}{*}{REQ.PE.1-C} & 1 
    & System should locate users within the building with an accuracy of 1m
    & Coordinates of the ID tags should be within a radial $\pm$1m of the actual location    \\ 
    \cline{2-4}
    & 2 
    & Approximate locations should be verified to be $\pm$1m of actual 
    & A set of automated tests is run to check if system calculates correct Coordinates 
    within $\pm$1m  \\
    \hline 
\end{tabular}
\caption{PoC Performance Requirement Test Plans}
\end{table}

\break
\subsubsection{Performance - Signal Test Plans}

\begin{table}[h!]
    \centering
    \begin{tabular}{|m{0.15\linewidth}|m{0.02\linewidth}|m{0.3\linewidth}|m{0.45\linewidth}|} 
    \hline
    % Change this line to match #. (Section) Test
    \multicolumn{4}{|l|}{Performance - Signal Tests }           \\ 
    % -----DONT NEED TO CHANGE THIS SECTION -----
    \hline
    REQ.ID      & \multicolumn{2}{l|}{Testing Criteria}      & Expected Output          \\ 
    \hline
    % ----DONT NEED  TO CHANGE THIS SECTION ----
    
    % Row of req ID, #, Testing Criteria, Expected Output
    REQ.PE.10 - C                 
    & 1 
    & 50\% of data transmitted via wireless communication must not be corrupted or lost
    & Arduino serial output should show data received/transmitted is at least 50\% correct    \\ 
    \hline    
    
    % Row of req ID, #, Testing Criteria, Expected Output
    \multirow{2}{*}{REQ.PE.11-C} 
    & 1 
    & 2.4 GHz wireless communications should be used to transfer data between beacons and ID tag
    & 2.4 GHz radio module transceiver for both the beacons and ID tag is communicating     \\ 
    \cline{2-4}
    & 2 
    & Beacons and ID tags only communicate to each other and other wifi/bluetooth signals should not interfere
    & Beacons and ID tags receive data from each other only and do not communicate with other signals \\
    \hline 
    
    % Row of req ID, #, Testing Criteria, Expected Output
    REQ.PE.12-C                  
    & 1 
    & Latency for wireless communication should be less than 100ms
    & Automated tests show signal transfer is less than 100ms    \\ 
    \hline    
\end{tabular}
\caption{PoC Performance - Signal Requirement Test Plans}
\end{table}

%\end{document}