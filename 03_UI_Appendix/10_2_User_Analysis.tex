

\subsection{User Analysis}
\medskip
The Akriveia Beacon system has three levels of targeted users: emergency first responders, IT or systems administrators, and company employees. Emergency first responders are considered to be the primary users, the system will provide high level layer interaction where only the most necessary elements are provided allowing intuitive access, decreasing the possibility of mistakes. The IT/system administrators are secondary users, who provides maintenance and upkeep of the system. Tasks such as managing ID tags, accounts, performing system wide maintenance, and perform potential upgrade, repair, and replacement procedures. Lastly, the tertiary users would be employees of the company using the Akriveia Beacon system. Employees would wear the ID tags alongside their everyday carry items such as mobile phones or access cards. The different levels of users will require different levels of interaction between them and the system. As such, each level of users will have requirements, access and control of the system. This section of the document will outline the different levels of user interactions, user background requirements, and detail some of the basic use cases for the primary, secondary, and tertiary users of the Akriveia Beacon system.

\bigskip
The primary users of the Akriveia system are the emergency first responders who are the first to arrive and provide assistance at the scene of an emergency, accident or disaster. First responders typically include paramedics, emergency medical technicians, police officers, firefighters, rescuers, and other trained professionals. In the intended situation where a search and rescue operation will be performed, the targeted primary user to be interacting with the system will be considered to be the person who is the top executive rank or commanding officer (Fire Chief) of the fire department on scene. The fire chief will cover the standard operating guidelines (SOGs) include basic communications with firefighter units deployed into buildings \cite{R10-2-1}. For the primary user, brief background knowledge on the operation of basic electronic equipment such as laptops and tablets are essential. Since user interaction between the primary user and the Akriveia Beacon system is through a graphical user interface hosted on basic electronic equipment, the user is required to have some form of familiarity with such devices. Once the system is incorporated more into current infrastructure, training could also be provided to the primary users if needed. Furthermore, basic comprehension of blueprint reading and ability to recognition and understanding of simple legends, icons, and other associated information on the UI is required. In addition, the primary users should have the grammatical prowess to understand the English language. Having fulfilled these user requirements, primary users can optimally benefit from the Akriveia Beacon system.

\bigskip
The secondary users of the Akriveia system are system administrators or IT technician that will be 
performing registration, and maintenance of the system. For secondary users, formal technical background is required; since the secondary users will be performing tasks such as installation and configuration of appropriate software and functions according to specifications. As well as to ensure the security and privacy of the networks and computing systems. Their primary tasks include managing ID tag accounts associated with employees, perform inspection of equipment such as beacons, ID tags, and data processing unit. As well as to ensure functionality of the system by enabling and operating the system during disaster drills. 

\bigskip
Lastly, the tertiary users are considered to be the employees of the company that will be incorporating the Akriveia beacon system into their infrastructure. The tertiary users require minimal interaction with the system during normal every day operation since they will only need to carry the ID tags on them while on company grounds. However, during emergencies or disaster, if needed the tertiary user must be able to enable the ID beacons to enable broadcasting of their location so that the primary users are able to locate them through the GUI provided by the system.




