

\subsection{User Analysis}
\medskip
The Akriveia Beacon system has two main targeted users: emergency first responders, and the IT or systems administrators. Emergency first responders are considered to be the primary users as the system is designed to operate under emergency situations involving these personnels. For Emergency first responders the system will provide a very high level layer of interaction, where only the most necessary interaction are provided allowing intuitive access and decreasing the probability for mistakes. The system/IT administrators, the secondary users, are the personnels that will provide maintenance and upkeep of the system. Tasks such as managing user accounts for ID tags, performing system wide maintenance for the beacons and data processing units, and any potential upgrade, repair, and replacement procedures will be performed by these users. 
\bigskip


\subsubsection{User - Emergency First Responders}
\medskip
The primary users of the Akriveia system are the emergency first responders who are the first to arrive and provide assistance at the scene of an emergency, such as an accident or disaster. First responders typically include paramedics, emergency medical technicians, police officers, fire-fighters, rescuers, and other trained members of organisations. In the intended situation where a search and rescue operation will be performed, the targeted primary user to be interacting with the system will be considered to be the person who is the top executive rank or commanding officer in a fire department, or the Fire Chief. The fire chief will cover the standard operating guidelines (SOGs) include basic communications with fire-fighter units deployed into buildings. 
\medskip
For the primary user, brief background knowledge on the operation of basic electronic equipment such as laptops and tablets are essential. Since the main user interaction between the primary user and the Akriveia Beacon system is through a graphical user interface hosted on these basic electronic equipment, the user is required to have some form of familiarity with the equipment. Once the system is incorporated more into current infrastructure, training could also be provided to the primary users. Furthermore, basic comprehension of blueprint reading and ability to recognition and understanding of simple legends, icons, and other associated information on the user interface is required. In addition, the primary users should have the grammatical prowess to understand the English language, as well as familiarity with handling electronic devices such as laptops and tablets. Having fulfilled these mentioned user requirements, the primary users can optimally benefit from the Akriveia Beacon system.
\bigskip


\subsubsection{User - IT/systems administrators}
\medskip
The secondary users of the Akriveia system are the system administrators or IT technician that will be 
performing registration, maintain and upkeep of the system. For secondary users, formal technical background is required to perform system maintenance and upgrades. The secondary users will perform installation and configure appropriate software and functions according to specifications. As well as to ensure the security and privacy of the networks and computing systems. Their primary tasks include managing ID tag accounts associated with employees, perform inspection of equipment such as beacons, ID tags, and data processing unit. As well as to ensure functionality of the system by enabling and operating the system during disaster drills. Operational training for secondary users will be provided if necessary.







