

\subsection{Technical Analysis}
\bigskip
This section will analyze the consideration for Seven Elements of UI Interface as outlined by Don Norman for the Akriveia Beacon system; which includes the following design factors, discoverability, feedback, conceptual models, affordances, signifiers, mappings, and constraints [10\_3\_1]. By incorporating these design element in to the system, the usability and quality of the final product can be substantially improved. 


\subsubsection{Discoverability}
\medskip
Discoverability: Is it possible to even figure out what actions are possible and where and how to perform them [10\_3\_1]?  In the context of product and interface design, discoverability is the degree of ease with which the user can find all the elements and features of a new system when they first encounter it [10\_3\_2]. The Akrivia beacon system is designed to operate under emergency situations and disasters, which means that it is paramount that the UI creates discoverability for its users. The overall interaction should be simple enough for each level of users to comprehend and understand without the need for much interpretation. 

\bigskip
Primary user - First Responders’ main point of interaction with the Akriveia Beacon system is through the GUI. The GUI will be simple with intuitive design allowing for quick understanding of the system and any relating concepts. For prototype UI design, the interface will be quick to access containing a scaled blueprint of the structure with colored indicators will convey the location of victims on the map view (similar to figure XXX). 

\bigskip
Secondary user - system and IT administrators will have more in depth access to the system. Since they are required to manage profiles related to each employee and their associated ID tag, the GUI needs to be simple and robust (see section 10.4 UI Mock-Ups). Actions such as adding, removing, and editing profiles or system configuration must be intuitive. 

\bigskip
Tertiary users - The ID tags are small in design and resembles and access card so the user know how to wear the device. In the case of an emergency the employees must toggle a simple push button to enable broadcasting of current location to the beacons. Otherwise the user must remember to keep ID tag on person while on company property.
\bigskip


\subsubsection{Feedback}
\medskip
Feedback - There is full and continuous information about the results of actions and the current state of the product or service. After an action has been executed, it is easy to determine the new state. As the main layer of interface between the users and the system, the GUI must provide visual indicators for any actions performed. Indicators such as confirmation messages and UI element state changes will be shown. Most importantly, while system is in active mode, icons and indicators on the map view must be updated in near real time to provide constant location data to the users. Furthermore, beacons and ID tags will also provide visual feedback to its users via simple LEDs indicate active or inactive system; as well as other information such as battery level or state of data transmission. 
\pagebreak


\subsubsection{Conceptual models}
\medskip



\subsubsection{Affordances}
\medskip



\subsubsection{Signifiers}
\medskip



\subsubsection{Mappings}
\medskip


\subsubsection{Constraint}
\medskip




