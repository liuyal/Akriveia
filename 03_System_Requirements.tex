

\setcounter{section}{2}
\section{System Requirements}
\bigskip
TRIWAVE SYSTEMS is dedicated to producing a high-quality product. To ensure production quality, various requirements at different stages of development will be stated clearing in this section. These requirements will be covering all aspects of the Akriveia Beacon technology. TRIWAVE SYSTEMS has decided that the following requirements, when met, will ensure that this product will be an asset to all first responders that use it, in as many emergency situations as possible. TRIWAVE SYSTEMS puts a high price on safety, reliability, and ease of use, so that this product may help save lives and keep first responders safe.  The functionalities that are labelled with a '\textbf{C}', as Proof-of-Concept, will be presented at the end of ENSC 405W.

\bigskip
\textit{Note: Requirements marked with ‘*’ are not carried over into subsequent development phases}


\bigskip
\subsection{General Requirements}
In order for the Akriveia Beacon system to be reliable, general high-level requirements describing the important core functions of the system are detailed in the table below. By following these requirements, TRIWAVE SYSTEMS ensure the overall system function and intended purpose are met at every phase of the development cycle.

\bigskip

\bgroup
\def\arraystretch{1.5}
\begin{table}[H]
\centering
\begin{tabular}{ | m{3.25cm} | m{12.5cm} |}
 \hline
 \textbf{REQ.GE.1 - C} & The system must be intended for indoor use only \\
\hline
 \textbf{REQ.GE.2 - C} & The system must have two modes of operation: idle mode and emergency mode \\
\hline
 \textbf{REQ.GE.3 - C} & The system must pinpoint ID tag location within one floor buildings \\
\hline
 \textbf{REQ.GE.4 - C} & Access to the system and system data must not be available to the general public \\
\hline
 \textbf{REQ.GE.5 - P} & First responders must only have access to the system during disaster situations \\
\hline
 \textbf{REQ.GE.6 - P} & The system must track more than one ID tag \\
\hline
 \textbf{REQ.GE.7 - F} & The system must locate ID tags in a building with multiple floors \\
\hline
 \textbf{REQ.GE.8 - F} & The system must remain operational after earthquakes with magnitude below 6.9 \\
\hline
 \textbf{REQ.GE.9 - F} & The system must remain operational after or during small scale fires \\
\hline
 \textbf{REQ.GE.10 - F} & The final marketed product consisting of one data processing unit, three beacons and one ID tag must not cost more than CAD \$200.00 \\
\hline
\end{tabular}
\caption{General Requirements}
\end{table}

\break
\subsection{Hardware Requirements}
Robust and reliable hardware will be absolutely necessary in any system that interacts with emergency and disaster management. TRIWAVE SYSTEMS has chosen to develop the proof-of-concept prototype using 2.4 GHz radio modules as feasibility testing, before upgrading the system to using Ultra Wide Band (UWB) modules for further prototyping. Micro-controller Units (MCUs) will be used to control the beacons and ID tags. The requirements below detail the requirement for hardware functionality of the Akrivia Beacon system.

\bigskip

\bgroup
\def\arraystretch{1.5}
\begin{table}[H]
\centering
\begin{tabular}{ | m{3.25cm} | m{12.5cm} |}
 \hline
 \textbf{REQ.HW.1 - C} & The data processing unit must use a \Gls{SBC} for data processing and display \\
\hline
 \textbf{REQ.HW.2 - C} & The beacons must use a MCU for intermediate data processing \\
\hline
 \textbf{REQ.HW.3 - C} & The ID tags must use a MCU for request processing \\
\hline
 \textbf{REQ.HW.4 - C} & Each MCU must use Serial Peripheral Interface (SPI) to communicate with transceiver \\
\hline
 \textbf{REQ.HW.5 - C*} & The beacons must use 2.4 GHz radio modules as transceivers* \\
\hline
 \textbf{REQ.HW.6 - C*} & The ID tags must use 2.4 GHz radio modules as transceivers*  \\
\hline
 \textbf{REQ.HW.7 - C*} & The beacons must communicate with the data processing unit via serial USB* \\
\hline
 \textbf{REQ.HW.8 - P} & The beacons must communicate with the data processing unit via a closed Wi-Fi network \\
\hline
 \textbf{REQ.HW.9 - P} & The beacons must use 3.5-6.5 GHz UWB radio modules as transceivers \\
\hline
 \textbf{REQ.HW.10 - P} & The ID tags must use 3.5-6.5 GHz UWB radio modules as transceivers \\
\hline
\end{tabular}
\caption{Hardware Requirements}
\end{table}

\break
\subsection{Electrical Requirements}
Since the system is designed for emergency disaster situations, it is crucial that sufficient power is provided to each device at any given time. In order for the system to be reliable and effective, the electrical systems must be robust and efficient. TRIWAVE SYSTEMS has compiled a strict set of electrical requirements that ensures the beacons and ID tags will operate in a safe and efficient way.

\bigskip

\bgroup
\def\arraystretch{1.5}
\begin{table}[H]
\centering
\begin{tabular}{ | m{3.25cm} | m{12.5cm} |}
 \hline
 \textbf{REQ.EC.1 - C} & Each beacon shall be powered through standard North American power outlets (120V AC, 60Hz, type A/B) \\
\hline
 \textbf{REQ.EC.2 - C} & Each ID tag must have its own 3.3 V battery power source \\
\hline
 \textbf{REQ.EC.3 - P} & Each ID tag must have an on-off toggle switch  \\
\hline
 \textbf{REQ.EC.4 - P} &  Each ID tag must remain in an idle, low power state with no more than 5W of power consumption while not in a disaster  \\
\hline
 \textbf{REQ.EC.5 - P} &  Each beacon must remain in an idle, low power state with no more than 5W of power consumption while not in a disaster \\
\hline
 \textbf{REQ.EC.6 - P} & Each beacon shall have its own backup power source (9V battery)  \\
\hline
 \textbf{REQ.EC.7 - P} & Each ID tag will use RF power harvesting technology to maintain charge on its batteries \\
\hline
 \textbf{REQ.EC.8 - F} &  Each ID tag must have a primary switch using an additional RF harvester as the trigger and a secondary manual switch to turn device on only   \\
\hline
 \textbf{REQ.EC.9 - F} & Each beacon shall incorporate a power caster device \\
\hline
 \textbf{REQ.EC.10 - F} & The data processing unit must have a battery backup (UPS) \\
\hline
 \textbf{REQ.EC.11 - F} & Users must no longer be capable of turning off the ID tag power switch manually \\
\hline
\end{tabular}
\caption{Electrical Requirements}
\end{table}

\break
\subsection{Software Requirements}
\medskip
The Akriveia system will be composed of an intricate software stack containing a database and a web server hosted GUI. The data processing unit is the main computational unit and will be implemented with trilateration algorithms to locate ID tag positions in near real time. In order for the software system to be reliable, secure and accurate the following requirements were made.

\bigskip

\bgroup
\def\arraystretch{1.5}
\begin{longtable}[H]{ | m{3.5cm} | m{12.5cm} |}
 \hline
 \textbf{REQ.SW.1 - C} & The system must use time of flight to locate ID tags \\
\hline
 \textbf{REQ.SW.2 - P*} & The system must use 2D trilateration methods to locate ID tags* \\
\hline
 \textbf{REQ.SW.3 - P} & Distance calculations must be performed using RSSI data received from beacons \\
\hline
 \textbf{REQ.SW.4 - P} & Admin shall have access the the data processing unit through credential verification system \\
\hline
 \textbf{REQ.SW.5 - P} & Admins must be able to create employees account associated with each ID tag \\
\hline
 \textbf{REQ.SW.6 - P} & First responder ID tags must be assicated with a pre-defined first responder account \\
\hline
 \textbf{REQ.SW.7 - P} & The data processing unit must be able to periodically check ID tags to verify tags are in working condition (correctness of data, device not broken) \\
\hline
 \textbf{REQ.SW.8 - P} & The system must detect and report ID tag defects, if any \\
\hline
 \textbf{REQ.SW.9 - P} & The data processing unit will store location history through out the event of an emergency \\
\hline
 \textbf{REQ.SW.10 - F} & Scaled blueprints of the monitor area must be able to be uploaded to the data processing unit for accurate layout and location \\
\hline
 \textbf{REQ.SW.11 - F} & The ID tags broadcasting must be able to be turned off from the servers \\
\hline
 \textbf{REQ.SW.12 - F} & The data processing unit must distinguish between different floors and be able to provide locations of ID tags for any floor serviced at any time during an emergency \\
\hline
 \textbf{REQ.SW.13 - F} & The system will provide floor plans for operators to track locations of disaster victims \\
\hline
 \textbf{REQ.SW.14 - F} & The system must use 3D trilateration methods to locate ID tags \\
\hline
 \textbf{REQ.SW.15 - F} & The system must use a closed Wi-Fi network for data forwarding from Beacon to data processing unit \\
\hline
\caption{Software Requirements}
\end{longtable}

\break
\subsubsection{Software - UI Requirements}
\medskip
Emergency first responders will be the first to interact with the Akriveia beacon system during a disaster. As such, the user interface must be intuitive and quick to use in a highly chaotic environment, when time is precious. The following table highlights important user interface requirements of the system.
\bigskip

\bgroup
\def\arraystretch{1.5}
\begin{longtable}[H]{ | m{3.5cm} | m{12.5cm} |}
\hline
 \textbf{REQ.SW.15 - P} & The \Gls{UI} must be intuitive to use for operators \\
\hline
 \textbf{REQ.SW.16 - P*} & The system must display a simple floor-plan/blueprint represented as simple geometry (ie. rectangle)* \\
\hline
 \textbf{REQ.SW.17 - P} & The system UI must display location of ID tags in near real time \\
\hline
 \textbf{REQ.SW.18 - P} & Each ID tags must be identified on the UI map by distinct symbols and/or tags \\
\hline
 \textbf{REQ.SW.19 - P} & The system UI application must fill the entire screen but not obscure any system-wide status bars. \\
\hline
 \textbf{REQ.SW.20 - P} & UI tag for rescue operations must have distinct indication from civilian ID tags \\
\hline
 \textbf{REQ.SW.21 - F} & The System UI must support mobile devices such as laptop PCs and \Gls{iOS} tablets \\
\hline
 \textbf{REQ.SW.22 - F} & The UI must be able to display blueprints and location markers for all serviced floors in building \\
\hline
\caption{Software - UI Requirements}
\end{longtable}
\break

\subsection{Performance Requirements}
\medskip
Reliability and accuracy are the most important attributes of the Akriveia system as it is a system designed to be widely used by first responders during emergency situations. Performance of the product maintains top priority before the system reaches general availability. The requirements below reflect the importance TRIWAVE SYSTEMS puts on the performance of the Akriveia Beacon system.

\bigskip
\bgroup
\def\arraystretch{1.5}
\begin{table}[H]
\centering
\begin{tabular}{ | m{3.5cm} | m{12.5cm} |}
 \hline
 \textbf{REQ.PE.1 - C*} & The system shall locate users within the building with an accuracy of 1m* \\
\hline
 \textbf{REQ.PE.2 - P} & The system shall track each ID tag in near real time with a latency of no more than 3 seconds \\
\hline
 \textbf{REQ.PE.3 - P} & The system must undergo a calibration process on start-up before usage \\
\hline
 \textbf{REQ.PE.4 - P} & The data processing unit must restart within 120 seconds \\
\hline
 \textbf{REQ.PE.4 - P} & The data processing unit must be fully functional within 120 seconds after booting \\
\hline
 \textbf{REQ.PE.4 - P} & The system UI must be accessible with in 2 seconds when operators directly connect to the system data processing unit \\
\hline
 \textbf{REQ.PE.5 - F} & The system shall locate users within the building with an accuracy of 0.5m \\
\hline
 \textbf{REQ.PE.6 - F} & The data processing unit's location processing throughput must be at least 100 employee locations per second \\
\hline
 \textbf{REQ.PE.7 - F} & Each beacon must be capable of detecting at least 100 employees within a 100m radius per second \\
\hline
 \textbf{REQ.PE.8 - F} & The ID Tags must remain operational in temperatures less than 60 degrees Celsius \\
\hline
 \textbf{REQ.PE.9 - F} & The Beacons must remain operational in temperatures less than 60 degrees Celsius \\
\hline
\end{tabular}
\caption{Performance Requirements}
\end{table}

\break

\subsubsection{Performance Requirements - Signal}
\medskip
In addition to basic performance requirement, the Akriveia beacon heavily relies on wireless signal communications. The following table details the requirements for signal communication and processing.
\bigskip

\bgroup
\def\arraystretch{1.5}
\begin{table}[H]
\centering
\begin{tabular}{ | m{3.5cm} | m{12.5cm} |}
\hline
 \textbf{REQ.PE.10 - C*} & 50\% of data transmitted via wireless communication must not be lost or corrupted* \\
\hline
 \textbf{REQ.PE.11 - C*} & Wireless communication must be done using 2.4 GHz wireless frequencies* \\
\hline
 \textbf{REQ.PE.12 - C*} & Wireless communication between transceivers must not have a latency of more than 100ms* \\
\hline
 \textbf{REQ.PE.13 - P} & The latency of wireless communication between transceivers must be less than 50ms \\
\hline
 \textbf{REQ.PE.14 - P} & Wireless communication must be established on 3.5-6.5 GHz UWB frequencies \\
\hline
 \textbf{REQ.PE.15 - F} & 5\% of data transmitted via wireless communication must not be lost or corrupted \\

\hline
\end{tabular}
\caption{Performance - Signal Requirements }
\end{table}
