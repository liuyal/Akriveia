

\setcounter{section}{2}
\section{Current Market \& Competitions}
\bigskip



\subsection{Competition}
\medskip
There are a number of competitors in the market that uses the same UWB technology for indoor localization as Triwave Systems. However, the application of their products is mainly associated with tracking equipments and items in warehouse or production facilities, which differs from the scope of search and rescue and does not compete directly with our product. 

\bigskip
infsoft is a competitor that provides indoor tracking solution mainly to industrial environments using UWB sensors. Similar to Triwave Systems, their tracking system consists of two components: locator tags and locator nodes, for which the locator tags can be attached to people or objects and the locator nodes are installed at fixed locations in an infrastructure. Some suggested uses cases for their UWB locating systems are forklift localization, route analysis, asset tracking and tugger train management, which are mainly in warehouse or production facilties \cite{R3-1}. In applications where locator tags are attached to immobile objects such as asset tracking, such a system would be beneficial as it may prevent incorrect inventories. However, in applications where the locator tags are attached to people such as search and rescue, privacy becomes an issue as their navigation within the coverage area of the locator nodes are exposed in real time. Triwave Systems ensures that user privacy is preserved in all non emergency situations by preventing ID tags to broadcast signals which forbids data reception from our beacons. This solution also decreases power consumption of ID tags significantly, hence lengthening their battery life as well. 

\bigskip
Pozyx is another company that provides tracking solutions using UWB technology for production and other industries. With the same system components as of infsoft, the company categorizes its products in two families: Creator and Enterprise. The Creator kit pricing at 1050 Euros consists of 4 tags and 5 anchors is aimed for hobbyist prototyping purposes, whereas the Enterprise kit pricing at 3900 Euros consists of 3 tags and 6 anchors is used in large industrial environments \cite{R3-2}. In terms of accessibility, both product families may not be ideal for their dedicated comsumers due to their high price points. Despite difference in use case, Triwave System integrates components that are less costly to build ID tags and beacons in order to maintain a widely accessible yet cost effective system. 

\bigskip
Instead of tags and anchors, the wireless sensor invented by the King Abdullah University of Science and Technology is a handheld portable device engineered to detect human body movements such as respiratory chest movements with UWB technology. It is low cost solution with high level of accuracy and the ability to penetrate obstacles due to benefits of UWB technology, which is seemingly ideal for search and rescue \cite{R3-3}. However, problems arise due to the handheld nature of the device in search and rescue. Firstly, the chance of locating victims is highly dependent on the coverage area and direction of the search conducted by the device user. This human dependency may pose limitations to its effectiveness on search and rescue since it is difficult to have full coverage on a disaster site. Lastly, having an extra device to carry can also be an extra burden to first responders who carries gears and equipment during a search operation. Triwave Systems solves these problems by having UWB sensors attached to potential victims instead of first responders. Search coverage is also handled by the distribution of our beacons and our data processing unit allows first responders to locate victims more effectively without carrying a device into dangerous environments.


