

\subsection{Empirical Usability Testing}
\medskip
This section details the completed empirical usability testing with users and outlines the methods of testing required for future implementations. Empirical usability testing will be carried out by the engineers at TRIWAVE SYSTEMS to systematically determine the usability of the user interface design of the Akriveia Beacon System. 

\bigskip
During empirical usability testing, testing will be carried out in cycles with real users consists of volunteer participants. The first cycle occurs near the end of the Prototype phase and the second cycle occurs near the end of the Final Product phase. Testing will be done with two small groups of participants that are unfamiliar with project development environment. First group will be asked to perform usability test cases outlined in Appendix A. An observer will document actions and observations of the testing process as well as to keep note of average time to complete each task, the amount of errors and error rate, number of tasks completed, and perform a sequence analysis. Issues will be represented similar to the method mentions in 10.6 Analytical usability testing. With the collected data the designers will re-evaluate the user interface for possible solutions for issues. After re-design and implementation a second small group of participants will be asked to perform the same tasks as the first group

\bigskip
From the results generated by participants the following usability elements will be addressed throughout the two testing cycles and development stages. 

\medskip
\begin{itemize}
\setlength\itemsep{0.1mm}
	\item \textbf{Easability:} The familiarity and intuitiveness of the system and how comfortable the users are with the user interfaces in general. 
	\item \textbf{Navigation:} The reliability of the navigation sequences are, how easy is it for the users to understand paths, and/or short cuts. Can the users easily retrace their steps or go back to previous states if they have made a mistake?
	\item \textbf{Responsiveness:} Does the users receive sufficient feedback from interacting with the system? 
	\item \textbf{Intuitiveness:} How quickly can a new user familiarize themselves with the user interface? Whether or not the users are able to perform tasks within a certain amount of time?
	\item \textbf{Robustness:} Safety and reliability of the device and system are addressed by eliminating or minimizing potential error (slips and mistakes) and enabling error recovery.
\end{itemize}

\medskip
By following these usability testing methods mentioned above for the Akriveia Beacon, the engineers and designers at TRIWAVES SYSTEMS can ensure a reliable and intuitive user interface will be produced to meet the needs of its end users. 
