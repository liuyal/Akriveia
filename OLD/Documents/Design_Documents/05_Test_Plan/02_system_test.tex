

\setcounter{section}{1}
\section{System Level Test Plan}
\bigskip

The Akriveia Beacon consists of various hardware, electrical and software components. As a system that will be operating under environments that dangerous during emergency disasters, the system must be tested  rigorously to ensure minimal error or failure rate. Each major components must be tested throughput individually and together as an integrated system. Below are test cases detailing the steps and procedures of conducting system level tests for the Akriveia beacon to ensure that its main functionalities are working as expected and meets specified requirements. 

\bgroup
\def\arraystretch{1.2}
\begin{table}[h!]
    \centering
    \begin{tabular}{|p{0.07\linewidth}|p{0.45\linewidth}|p{0.40\linewidth}|} 
    \hline
    ID & Test Procedure & Validation \\ 
    
    \hline
    SL-01
    & Send start tracking command to beacon system from web server UI. ID tag is turned on into tracking mode.  
    & All active beacons receive and return acknowledge  back to server. Tags are ranged with correct distance values received by server from all beacons. \\ 
    \hline    

    \hline
    SL-02
    & Send end tracking command to beacon system from web server user interface. ID tag is turned off.
    & All active beacons receive and return acknowledge back to server. Server stops receiving tag ranging values from beacons.   \\ 
    \hline   
    
    \hline
    SL-03
    & System is set in tracking mode (steps in SL-01). Tag ranging is verified by physically measuring distance between tag and beacon with tape measurer 
    & Compare physical measured distance to beacon ranging values received on web server. The difference must not exceed 0.5m as specified in the design requirements.  \\ 
    \hline   
    
    \hline
    SL-04
    & System is set in tracking mode (steps in SL-01). ID tag is turned on into tracking mode. Tester must wear an ID tag and be able to walk freely in test area to validate trilateration algorithm. 
    & Verify Trilateration method is implemented properly. Tracking resulting in correct coordinates when personnel wearing id tag is moving around the test area. Difference in coordinates must not exceed 0.5m. \\ 
    \hline
    
    \hline
    SL-05
    & System is set in tracking mode (steps in SL-01). Multiple ID tags (at least 2) are present in the test area and are turned on in tracking mode. 
    & Web server receives proper beacon messages. Each individual id tag can be ranged correctly and trilateration calculation yield result with tolerance less than 0.5m.  \\ 
    \hline
    
    \hline
    SL-06
    & System is set in tracking mode (steps in SL-01). At least 4 beacons must be present and registered in the web server. At least 1 ID tag is present in the test area. During tracking mode power off one of the beacons to simulate beacon failure. 
    & Before beacon failure tracking shows correct results. After beacon failure web server shows which beacon(s) is down and tracking should still return correct ranging results if there are a least 3 beacons operational.  \\ 
    
    \hline
    SL-07
    & Web server is started on server machine hosted on public IP and local IP and  is accessible locally and remotely from another device (laptop, tablet, etc.)
    & System configuration can be performed correctly. Operations such as adding/editing/removing beacons, users, maps through the web server user interface can be performed with correct results\\ 

    \hline
    \end{tabular}
    \caption{System Level Test Cases}
\end{table}















