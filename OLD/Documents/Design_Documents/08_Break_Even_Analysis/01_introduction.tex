

\setcounter{section}{0}
\section{Introduction}

\bigskip
Over the last couple of decades urban centers around the world have faced substantial population growth. As a result, the number of larger and more complex commercial structures in core urban centers around the world is rapidly increasing. With a large population and massively complex buildings in relatively small areas, the potential impact of damages and casualties in the event of a disaster have greatly increased as well. Due to increased urbanization and complexity of urban structures, search and rescue operations in indoor urban environments face various complications and uncertainties. According to Statistics Canada, an average of 135 fire related deaths occur each year from 2010 to 2014 [1]. 

\bigskip
As a solution to these major issues the Akriveia Beacon created by TRIWAVE SYSTEMS focuses on improving the locating and rescue process of personnel trapped in buildings during or after small scale disasters such as fires and low magnitude earthquakes, therefore minimizing potential damages and casualties. One great benefit to the Akriveia Beacon product is that it’s uniquely placed between two substantially large markets: The Search and Rescue equipment (SAR) market and the Global Indoor Location (GIL) market. SAR equipment is an old existing market and encompasses emergency vehicles such as trucks or helicopters to rescue responders devices such as the Akriveia Beacons. Estimated to be a \$113.62 billion dollars industry in 2017, the SAR market is expected to reach \$125.66 billion in 2022 [2]. On the other hand, GIL is an emerging market after the proliferation of smartphone users but also the ineffectiveness of GPS for indoor tracking. Currently the GIL market is at \$3.43 billion in 2015 and is expected to surpass \$29.4 billion in 2022 [3]. 

\bigskip
Inorder for the Akrivia Beacon to be a viable and economically sustainable product and for TRIWAVE SYSTEMS to become a successful company, it is important to consider engineering and economical cost of the product.  This will allow for the company to formulate a revenue plan that will cover the costs of operating a business. This Break-even analysis entails the calculation and examination of the margin of safety for the Akriveia beacon on the revenues collected and associated costs. By analyzing different price levels relating to various levels of demand, break-even analysis can determine the level of sales necessary to cover TRIWAVE SYSTEMS’ total fixed costs.
