
\setcounter{section}{8}
\section{Appendix A: Supporting Test Plans}
\bigskip


All test cases for each of the four test plans follows the below format. These test cases are derived from the 
requirements of the \textbf{\textit{Design Specifications}}.
\medskip
\begin{center}
	\textbf{[ TEST-\#]} 
\end{center}


\bgroup
\def\arraystretch{1.5}
\begin{table}[H]
\centering
\begin{tabular}{ | m{1cm} | m{13cm}| } 
\hline
\rowcolor{lightgray} \textbf{Code} & \textbf{Definition} \\ 
\hline
 \textbf{TEST} & Test Plan abbreviation. \\ 
\hline
 \textbf{\#} & Test Case ID \\ 
\hline
\end{tabular}
\caption{Test Case Encoding}
\end{table}

\bgroup
\def\arraystretch{1.5}
\begin{table}[H]
\centering
\begin{tabular}{ | m{7cm} | m{7cm}| } 
\hline
\rowcolor{lightgray} \textbf{Planning Stage} & \textbf{Abbreviation Code} \\ 
\hline
 Proof of Concept & C\\ 
\hline
 Prototype & P\\ 
\hline
 Final Product & F\\  
\hline
 Usability & U\\ 
\hline
\end{tabular}
\caption{Planning Stage Abbreviation Code}
\end{table}

Each test case is grouped into these four sections and their IDs start with a letter representing the group. These letters are shown in bold above.
The \textbf{\textit{Design Specifications}} test plans consists of Test cases with expected results. These test cases have been modified from the
previous test plans from the \textbf{\textit{Requirements Specifications}} to reflect current project progress.


%----------------------------------PoC Test plan ----------------------------------------------------------------------%
\pagebreak

\subsection{PoC Test Plan}
\medskip
% What testing is there supposed to be?
The Proof of Concept (PoC) Test Plan includes testing procedures for verifying and validating the \textbf{\textit{Design Specifications}} under a formal test environment. The goal of the PoC Test Plan is to ensure the basic requirements stated in the \textbf{\textit{Requirement Specifications}} are refined and developed into test cases. Three main goals for the PoC testing are as follows:
\begin{enumerate}
    \item 2.4 GHz chips are able to receive and transmit data accurately
    \item ESP32 (MCU) receives the transmission data 
    \item Raspberry Pi receives data serially from ESP32 MCU
\end{enumerate}

\begin{table}[h!]
    \centering
    
    \begin{tabular}{|m{0.05\linewidth}|m{0.45\linewidth}|m{0.45\linewidth}|} 
    \hline
    \multicolumn{3}{|l|}{\textbf{PoC Test Plan}}  \\ 
    \hline
    ID & Test Cases & Expected Results\\ 
    \hline
    
    % ESP32 bluetooth modules detects bluetooth servers in the proximity 
    % Console Serial output shows all nearby bluetooth devices with MAC addresses
    P-01
    & ESP32 bluetooth modules detects bluetooth servers in the proximity.  
    & Console Serial output shows all nearby bluetooth devices with MAC addresses. \\ 
    \hline    

    % ESP32 uses the RSSI formula to output values of nearby bluetooth devices
    % Arduino serial output is displaying RSSI values of surrounding bluetooth devices
    P-02
    & ESP32 uses the RSSI formula to output values of nearby bluetooth devices.
    & ESP serial out displays RSSI of surrounding bluetooth devices by MAC addresses. \\ 
    \hline

    % RSSI measurements varies at different distances
    % Take RSSI measurements at 1m; RSSI measurements should be higher at 2m; RSSI measurements should be lower at 0.5m
    P-03
    & RSSI measurements on serial output should vary at different distances.
    & RSSI measurements at 0.1m, 0.5m, 1m and 2m. The RSSI measurements should become more negative as distance increases.\\ 
    \hline

    % Measure RSSI measurements at 1m to get measured power
    % Log average of RSSI at 1m at different locations: empty room, busy hallway, LAB01 to get measured power
    P-04
    & Measure RSSI measurements at 1m to get measured power
    & Log the average RSSI at 1m at different locations. The average measurements at each location is the measured power at specific location. \\ 
    \hline
    
    % ESP32 beacon should take only ESP32 ID tag MAC address
    % Serial output should output RSSI of ID tags; Filter out all RSSI measurements other than ID tags
    P-05
    & ESP32 beacon should only output ESP32 ID tag MAC address.
    & Serial output should output RSSI of ID tags. Beacon programming filters out all RSSI measurements besides ID tags.\\ 
    \hline

    % ESP32 sleep mode activates after 2min period
    % ESP32 shuts off main processor and peripherals. MCU consumes only 10uA in sleep mode.
    P-06
    & ESP32 deep sleep mode activates after 2min period of inactivity. 
    & ESP32 shuts off main processor and peripherals. MCU consumes only 10uA in sleep mode.\\ 
    \hline

    % Raspberry PI transfers data through serial to a ESP32 beacon
    % "Start" command is sent serially by the Raspberry PI and received serially by the ESP32 beacon
    P-07
    & DPU transfers data through serial to a ESP32 beacon.
    & "Start" command sent serially by the DPU and received serially by the ESP32 beacon.\\ 
    \hline

    % Raspberry Pi (DPU) should receive RSSI measurements through serial communication from all 3 ESP32 beacons
    % Console output of the received RSSI measurements from 3 beacons
    P-08
    & Raspberry Pi (DPU) should receive RSSI measurements through serial communication from all 3 ESP32 beacons
    & Console output of the received RSSI measurements from 3 beacons. \\ 
    \hline

    % RSSI measurements are taken from all 3 beacons and converted into distances by Distance-RSSI formula
    % Console output of approximate distances of the 3 beacons to the ID tag
    P-09
    & RSSI measurements are taken from all 3 beacons and converted into distances by Distance-RSSI formula
    & Console output of approximate distances of the 3 beacons to the ID tag. \\ 
    \hline
    \end{tabular}
    \caption{PoC System (General) Test Plans - Part 1}
\end{table}
\pagebreak
\begin{table}[h!]
    \centering
    
    \begin{tabular}{|m{0.05\linewidth}|m{0.45\linewidth}|m{0.45\linewidth}|} 
    \hline
    \multicolumn{3}{|l|}{\textbf{PoC Test Plan}}  \\ 
    \hline
    ID & Test Cases & Expected Results\\ 
    \hline
    
    % Distances from the Distance-RSSI formula are used in the Trilateration algorithm to calculate approximate location
    % Location of the ID tag is given as (x, y) coordinates with (0, 0) being the location of the bottom left beacon
    P-10
    &  Distances from the Distance-RSSI formula are used in the Trilateration algorithm to calculate approximate location
    &  Location of the ID tag is given as (x, y) coordinates with (0, 0) being the location of the bottom left beacon. \\ 
    \hline

    % Real-time tracking of the ID tag is calculated and displayed on the Raspberry Pi console output
    % Location of the ID tags in (x, y) are constantly displayed on the serial output of the Raspberry Pi
    P-11
    & Real-time tracking of the ID tag is calculated and displayed on the Raspberry Pi console output
    & Location of the ID tags in (x, y) are constantly displayed on the serial output of DPU. \\ 
    \hline

    \end{tabular}
    \caption{PoC System (General) Test Plans - Part 2}
\end{table}

%----------------------------------Prototype Test plan ----------------------------------------------------------------------%
\pagebreak
\subsection{Prototype Test Plan}
\medskip
% What will the Prototype Test Plan include?
The Prototype Test Plan includes testing procedures for verifying and validating the \textbf{\textit{Design Specifications}} under formal test environment. The goal of the Prototype Test Plan is to ensure the requirements stated in the  \textbf{\textit{Requirement Specifications}} are meet. Three main goals for the Prototype testing are as follows:
\begin{enumerate}
    \item UWB chips are able to receive and transmit data accurately
    \item ESP32(MCU) integrates with the UWB modules and receives transmission data 
    \item Raspberry Pi (DPU) calculates the location coordinates from Time of Flight (ToF)
\end{enumerate}


\begin{table}[h!]
    \centering
    \begin{tabular}{|m{0.05\linewidth}|m{0.45\linewidth}|m{0.45\linewidth}|} 
    \hline
    \multicolumn{3}{|l|}{\textbf{Prototype Test Plan}}           \\ 
    % -----DONT NEED TO CHANGE THIS SECTION -----
    \hline
    ID & Testing Criteria & Observations       \\ 
    \hline
    % ----DONT NEED  TO CHANGE THIS SECTION ----
    
    % UWB chip and breakout board are soldered properly.  SPI Bus Pins are operating as expected
    % MISO/MOSI pins are outputting expected values. SCLK is receiving proper clock signals. 
    % IRQ pin is generating interrupts and CS pin remains at the correct state (active-low)
    T-01 
    & UWB chip and breakout board are soldered properly. SPI Bus Pins are operating as expected
    & MISO/MOSI pins are outputting expected values. SCLK is receiving proper clock signals. 
    IRQ pin is generating interrupts and CS pin remains at the correct active-low state.\\ 
    \hline
    
    % UWB chip and breakout board are soldered properly. Other Pins are operating as expected
    % EXTON tells external devices to go to sleep mode. GPIO7 general purpose I/O pin transmits data properly
    % RSTn pin resets Decawave 1000 chip. WAKEUP pin asserted to wake up DW1000 chip from sleep
    T-02
    & UWB chip and breakout board are soldered properly. Pins are operating as expected
    & Device sleep activate by EXTON. GPIO7 transmits data. RSTn pin resets DWM1000. WAKEUP pin trigger DW1000 to wake. \\ 
    \hline

    % ESP32 (MCU) transfers data to/fro from the UWB module based on SPI communications.
    % ESP32s should receive expected data from SPI-bus and output onto console output. Check pins are functioning as expected
    T-03 
    & ESP32 transfers data from the UWB module via SPI communications.
    & ESP32 receive expected data from SPI and output to console. pins are functioning as expected. \\ 
    \hline

    % UWB chips receive and transmit simple messages to and from other UWB chips through UWB frequencies of 3.4Ghz - 6Ghz
    % Received data is shown serially on the consoles for the chips
    T-04 
    & UWB chips receive and transmit simple messages to and from other UWB chips through UWB frequencies of 3.5-6.5 GHz.
    & Received data is shown serially on the consoles for the chips.   \\ 
    \hline

    % ESP32 MCUs are woken out of deep sleep by the GPIO # pin. ESP32 should wake the UWB through the WAKEUP pin
    % ESP32 MCUs and UWB are in active state; Start operating as programmed
    T-05
    & ESP32 are woken out of deep sleep by the GPIO \#26 pin. ESP32 should wake the UWB through the WAKEUP pin.
    & ESP32 and UWB are in active state; Start operating as programmed. \\ 
    \hline

    % ESP32 MCUs go into sleep after 2 minutues of inactivity. UWB module is sent to sleep though the EXTON pin.
    % ESP32 MCUs and UWB are in deep sleep mode
    T-06
    & ESP32 go into sleep after 2 minutues of inactivity. UWB module is sent to sleep though the EXTON pin.
    & ESP32 and UWB are in deep sleep mode. \\ 
    \hline

    % UWB ID tag receives "emergency" signal from UWB beacons. UWB chip ensure ESP32 MCU remains awake through the EXTON pin
    % ESP32 MCUs and UWB remains in active state. ESP32s are awake and do not go back to sleep
    T-07
    & ID tag receives "emergency" signal from beacons. UWB chip ensure ESP32 remains awake through the EXTON pin.
    & ESP32 and UWB remains in active state. ESP32s are awake and do not go back to sleep. \\ 
    \hline

    % The ID tags are able to be powered by the battery
    % ID tags are functional while being batter powered
    T-08
    & ID tags are to be powered by the battery.
    & ID tags are functional using battery power. \\ 
    \hline   
    
    % MCU receives timestamps from UWB and calculates ToF from sent and received data 
    % Serial output of ToF measurement. There should be manual verification of calculation
    T-09
    & MCU receives timestamps from UWB and calculates ToF from sent and received data.
    & Serial output of ToF measurement. There should be manual verification of calculation.   \\ 
    \hline
    
\end{tabular}
    \caption{Prototype Test Plans - Part 1}
\end{table}    
    
\pagebreak    
    
\begin{table}[h!]
    \centering
    \begin{tabular}{|m{0.05\linewidth}|m{0.45\linewidth}|m{0.45\linewidth}|} 
    \hline
    \multicolumn{3}{|l|}{\textbf{Prototype Test Plan}}           \\ 
    % -----DONT NEED TO CHANGE THIS SECTION -----
    \hline
    ID & Testing Criteria & Observations       \\ 
    \hline
    % ----DONT NEED  TO CHANGE THIS SECTION ----       
    
    % ToF measurements are taken and used in the Distance-ToF formulas to generate distance
    % Distances are displayed on the serial output. There should be manual verification of the calculations
    T-10
    & ToF are taken and used in the distance-ToF formulas to generate coordinates.
    & Distances are displayed on serial output. Require manual verification of calculations.  \\ 
    \hline    
    
    % Distances determined from ToF is sent to Raspberry Pi (DPU) for trilateration calculation
    % Serial output from Raspberry Pi showing the x-y coordinates from trilateration algorithm
    T-11
    & Distances estimation from ToF is sent to DPU for trilateration calculation
    & Serial output from Raspberry Pi showing the x-y coordinates from trilateration algorithm.    \\ 
    \hline
    
    % Raspberry Pi (DPU) should display x-y coordinates onto the webpage
    % Console output of the x-y coordinates of the ID tag in reference to the bottom left beacon as (0, 0)
    T-12
    & DPU should display x-y coordinates on webpage.
    & Console output of x-y coordinates of the ID tag in reference to the bottom left beacon as (0, 0).    \\ 
    \hline

    % Real-time tracking of the ID tag in x-y coordinates on the console
    % Webpage should show real-time changes in the x-y coordinates of the ID tag 
    T-13
    & Real-time tracking of the ID tag in x-y coordinates on the console
    & Webpage should show real-time changes in the x-y coordinates of the ID tag.       \\ 
    \hline 
    
    % Raspberry Pi (DPU) hosts the web server. Webpage can be accessed through Pi's wifi network
    % Devices can access the webpage through the devices' browser
    T-14
    & DPU hosts the web server. Webpage can be accessed through DPU's wifi network.
    & Devices can access the webpage browser.     \\ 
    \hline   

\end{tabular}
    \caption{Prototype Test Plans - Part 2}
\end{table}


%----------------------------------Final Product Test plan ----------------------------------------------------------------------%
\pagebreak
\subsection{Final Product Test Plan}
\medskip
% What will the Final Product Test Plan include?
The Final Product Test Plan includes all the testing procedures for verifying and validating the \textbf{\textit{Design Specifications}} under a formal
 test environment. The goal of the Final Product Test Plan is to ensure the basic requirements stated in the \textbf{\textit{Requirement Specifications}}
 are refined and developed into test cases. Three main goals for the Final Product testing are as follows:
\begin{enumerate}
    \item 2.4 GHz chips are able to receive and transmit data accurately
    \item Arduino micro-controllers(MCU) receives the transmission data 
    \item Raspberry Pi receives data serially from Arduino M
\end{enumerate}


\begin{table}[h!]
    \centering
    
    \begin{tabular}{|m{0.05\linewidth}|m{0.45\linewidth}|m{0.45\linewidth}|} 
    \hline
    \multicolumn{3}{|l|}{\textbf{Final Product Test Plan}}      \\ 
    % -----DONT NEED TO CHANGE THIS SECTION -----
    \hline
    ID   & Testcases    & Observations     \\ 
    \hline
    % ----DONT NEED  TO CHANGE THIS SECTION ----
    
    % Test for the accuracy of UWB ToF to ensure accuracy is within <1m.
    % Run a set of automated test for different distances to test the determined
    % ToF distances is accurate or within error range of 1m
    F-01
    & Test for the accuracy of UWB ToF to ensure accuracy is within $\leq$1m.
    & Different distances set up to test ToF estimation accurate within error range of 1m.  \\ 
    \hline
    
    % Emergency signal initiates emergency mode for all beacons & ID tags
    % All beacons are now in emergency mode. ID tags woken up will go into emergency state and stay woken up
    F-02 
    & Emergency signal initiates emergency mode for all beacons and ID tags.
    & All beacons are now in emergency mode. ID tags woken up will go into emergency state and stay woken up.  \\ 
    \hline
    
    % Beacons forward data from farthest ID tags to the Raspberry Pi (DPU) so that all ID tags 
    % can be displayed on the map
    % All ID tags distances are received by the DPU.
    F-03 
    & Beacons forward data from farthest ID tags to the Raspberry Pi (DPU) so that all ID tags 
    can be displayed on the map. & All ID tags distances are received by the DPU.  \\ 
    \hline

    % ID tags should drain minimal amount of power (<10uA) when in deep sleep mode 
    % Measured current drawn by ID tag in deep sleep is less than 10uA
    F-04
    & ID tags should drain minimal amount of power ($\leq$10uA) when in deep sleep mode.
    & Measured current drawn by ID tag in deep sleep is less than 10uA.  \\ 
    \hline

    % ID tags are woken up by a button press and checks for emergency signals from beacons
    % ID tags should go into emergency mode if beacons are sending emergency signals. Otherwise,
    % ID tags should go into deep sleep mode after 2 minutes of inactivity
    F-05
    & ID tags are woken up by a button press and checks for emergency signals from beacons.
    & ID tags should go into emergency mode if beacons are sending emergency signals. Otherwise, 
    ID tags should go into deep sleep mode after 2 minutes of inactivity  \\ 
    \hline

    % Beacon uses UDP protocol for communication to send out packets of data continuously without handshaking.
    % Beacons should continue to send and receive timestamped packets of data.
    F-06
    & Beacon uses UDP protocol for communication to send out packets of data continuously without handshaking.
    & Beacons should continue to send and receive timestamped packets of data.  \\ 
    \hline

    % The ID tag's components, ESP32 wifi and MCU module, UWB module and housing LED, are all powered by the battery.
    % All 3 components are operational and can all be powered by the battery.
    F-07 
    & The ID tag's components, ESP32 wifi and MCU module, UWB module and housing LED, are all powered by the battery.
    & All 3 components are operational and can all be powered by the battery.  \\ 
    \hline
    
    % The beacon's components, ESP32 wifi and MCU module, UWB module and housing LED, are powered through AC outlet.
    % All 3 components are operational and no issues occur.
    F-08 
    & The beacon's components, ESP32 wifi and MCU module, UWB module and housing LED, are powered through AC outlet.
    & All 3 components are operational and no issues occur.  \\ 
    \hline

\end{tabular}
    \caption{Final Product Test Plans - Part 1}
\end{table}

\pagebreak

\begin{table}[h!]
    \centering
    
    \begin{tabular}{|m{0.05\linewidth}|m{0.45\linewidth}|m{0.45\linewidth}|} 
    \hline
    \multicolumn{3}{|l|}{\textbf{Final Product Test Plan}}      \\ 
    % -----DONT NEED TO CHANGE THIS SECTION -----
    \hline
    ID   & Testcases    & Observations     \\ 
    \hline
    % ----DONT NEED  TO CHANGE THIS SECTION ----

	% All the components of the ID tag are integrated together: ESP32 (MCU) and wifi modules, UWB modules
    % , the battery and ID tag housing
    % ID tag is functional and all components are working together
    F-09
    & All the components of the ID tag are integrated together: ESP32 (MCU) and wifi modules, UWB modules
    , the battery and ID tag housing.
    & ID tag is functional and all components are working together.  \\ 
    \hline

    % Beacons use UDP protocol for communication to the Raspberry Pi (DPU) for real-time tracking.
    % Raspberry Pi updates the ID tags location in real-time to provide up-to-date tracking.
    F-10
    & Beacons use UDP protocol for communication to the Raspberry Pi (DPU) for real-time tracking.
    & Raspberry Pi updates the ID tags location in real-time to provide up-to-date tracking.  \\ 
    \hline

    % Beacons run on emergency battery power when A/C power supply is unavailable.
    % When power is disconnected, Beacons run no battery power until A/C power is available.
    F-11
    & Beacons run on emergency battery power when A/C power supply is unavailable.
    & When power is disconnected, Beacons run no battery power until A/C power is available.  \\ 
    \hline

% ---- Software test cases -------%

    % Raspberry Pi (DPU) displays the ID tag locations on a scaled map GUI.
    % All ID tags are shown on the scaled map with an accuracy of <1m.
    F-12
    & Raspberry Pi (DPU) displays the ID tag locations on a scaled map GUI.
    & All ID tags are shown on the scaled map with an accuracy of $\leq$1m.  \\ 
    \hline
    
    % Move the ID tags around the floor to check for real-time tracking.
    % GUI map shows ID tags current location with <5sec delay between the current and previous location.
    F-13
    & Move the ID tags around the floor to check for real-time tracking.
    & GUI map shows ID tags current location with 	$\leq$ 5s delay between location. \\ 
    \hline

    % Check if the MAC address and status in the GUI match with the devices
    % ID tags/Beacon statuses in the GUI are accurate and reflect the current situation
    F-14
    & Check if the MAC address and status in the GUI match with the devices.
    & ID tags/Beacon statuses in the GUI are accurate and reflect the current situation.  \\ 
    \hline

    % Test for any bugs or possible unauthorized access to data in the system.
    % Run a set of automated tests to look at any bugs or issues that arise. Manual
    % testing is required.
    F-15
    & Test for any bugs or possible unauthorized access to data in the system.
    & Automated tests to look for any bugs or issues that arise. Manual testing is required.  \\ 
    \hline


\end{tabular}
	\caption{Final Product Test Plans - Part 2}
\end{table}



%----------------------------------Usability Test plan ----------------------------------------------------------------------%
\pagebreak
\subsection{Usability Test Plan}
\medskip
% What will the usability test plan include
The Usability Test Plan includes all the testing procedures for verifying and validating the Usability Specifications under a 
formal test environment. The goal of the Usability Test Plan is to ensure the basic requirements stated in the 
\textbf{\textit{Appendix B: User Interface and Appearance}} section of the document are met. Three main goals for the Usability testing are as follows:
\begin{enumerate}
    \item ID tags and Beacons are intuintive to use and implement
    \item GUI is easy to navigate and access information for Primary Users
    \item GUI is easy to edit and customize configuration for Secondary Users
\end{enumerate} 

\begin{table}[h!]
    \centering
    \begin{tabular}{|m{0.05\linewidth}|m{0.45\linewidth}|m{0.45\linewidth}|} 
    \hline
    \multicolumn{3}{|l|}{\textbf{Usability Test Plan}}           \\ 
    % -----DONT NEED TO CHANGE THIS SECTION -----
    \hline
    ID    & Test Cases    & Observations     \\ 
    \hline
    % ----DONT NEED  TO CHANGE THIS SECTION ----
    
    %----Software side -------------------%
    % Connecting to the Beacons' wifi network should be simple and quick.
    % Select the network on the device's available wifi network. Input the password and the device should be connected.
    U-01 
    & Connecting to the Beacons' wifi network should be simple and quick.
    & Select the network on the device's available wifi network. Input the password and the device should be connected.  \\ 
    \hline   
    
    % Test if webpage address is correct and online.
    % Inputting the webpage address should bring up the login page for Triwave Systems Akriveia.
    U-02
    & Test if webpage address is correct and online.
    & Inputting the webpage address should bring up the login page for Triwave Systems Akriveia.   \\ 
    \hline
    
    % Test if the login credentials entered matches the one in the system database. Give an appropriate response.
    % Entering the incorrect login credentials gives an login error. Prompts the user to try again.
    % Entering the correct login credentials goes to the System Status or user homepage.
    U-03 
    & Test if the login credentials entered matches the one in the system database. Give an appropriate response.
    & Entering the incorrect login credentials gives an login error. Prompts the user to try again.
      Entering the correct login credentials goes to the System Status or user homepage.  \\ 
      
    \hline

    % Test if primary users sees only the Primary User Map View and Status View.
    % Primary user should not have access to any system configurations, adding map views or adding user views.
    U-04
    & Test if primary users sees only the Primary User Map View and Status View.
    & Primary user should not have access to any system configurations, adding map views or adding user views.   \\ 
    \hline

    % Test if secondary users can see Add Map View, Beacon View and Add User Views.
    % Primary user should have access to any system configurations, adding map views or adding user views.
    U-05
    & Test if secondary users can see Add Map View, Beacon View and Add User Views.
    & Secondary user should have access to any system configurations, adding map views or adding user views.   \\ 
    \hline
    
    % Adding maps, beacons or editing user information should be intuitive and simple.
    % IT personnel with minimal training can add maps, beacons or edit user information without frustration or difficulty.
    U-06
    & Adding maps, beacons or editing user information should be intuitive and simple.
    & IT personnel with minimal training can add maps, beacons or edit user information without frustration or difficulty.   \\ 
    \hline
   
    % Adding maps, beacons or editing user information should be intuitive and simple.
    % IT personnel with minimal training can add maps, beacons or edit user information without frustration or difficulty. 
    U-07
    & Installing new beacons should be intuintive and straight-forward.
    & Adding beacon page is easy to navigate and form is simple to fill out. IT personnel without any training 
    of Akriveia system should fill out the form without difficulty.   \\ 
    \hline

\end{tabular}
    \caption{PoC Software Requirement Test Plans - Part 1}
\end{table}

\pagebreak


\begin{table}[h!]
    \centering
    \begin{tabular}{|m{0.05\linewidth}|m{0.45\linewidth}|m{0.45\linewidth}|} 
    \hline
    \multicolumn{3}{|l|}{\textbf{Usability Test Plan}}           \\ 
    % -----DONT NEED TO CHANGE THIS SECTION -----
    \hline
    ID    & Test Cases    & Observations     \\ 
    \hline
    % ----DONT NEED  TO CHANGE THIS SECTION ----
    
    %------- Hardware Devices ----------%
    
    % ID tag design has clear indication where the button is pressed.
    % Users with no training understands intuitively where to press on the ID tag.
    U-08
    & ID tag design has clear indication where the button is pressed.
    & Users with no training understands intuitively where to press on the ID tag.  \\ 
    \hline

    % In emergency state, ID tag button when pressed will flash green continuously
    % ID Tags LED light will continue to flash green in emergencies
    U-09
    & In emergency state, ID tag button when pressed will flash green continuously.
    & ID Tags LED light will continue to flash green in emergencies.  \\ 
    \hline
    
    % ID tags LED flashes the correct light in non-emergency mode
    % LED will flash green for 1 sec if ID tag is operational and has no problems
    % LED will blink yellow for 3 sec if ID tag battery is low and need replacement.
    % LED will blink red for 3 sec if ID tag is not operational or has issues. Battery is on critical low power.
    U-10
    & ID tags LED flashes the correct light in non-emergency mode
    & LED will flash green for 1 sec if ID tag is operational and has no problems
      LED will blink yellow for 3 sec if ID tag battery is low and need replacement.
      LED will blink red for 3 sec if ID tag is not operational or has issues. Battery is on critical low power.  \\
    \hline

    % In emergency state, beacons will flash green continuously.
    % Beacons LED light will continue to flash green in emergencies
    U-11
    & In emergency state, beacons will flash green continuously.
    & Beacons LED light will continue to flash green in emergencies.  \\ 
    \hline

    % Beacons LED flashes the correct light in non-emergency mode
    % LED will flash green for 1 sec if Beacon is operational and has no problems
    % LED will blink yellow for 3 sec if Beacon has been reset. 
    % LED will blink red for 3 sec if ID tag is not operational or has issues.
    U-12
    & Beacons LED flashes the correct light in non-emergency mode
    & LED will flash green for 1 sec if Beacon is operational and has no problems
      LED will blink yellow for 3 sec if Beacon has been reset.
      LED will blink red for 3 sec if ID tag is not operational or has issues.  \\ 
    \hline

\end{tabular}
    \caption{PoC Software Requirement Test Plans - Part 2}
\end{table}

