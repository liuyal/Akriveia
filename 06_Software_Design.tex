

\setcounter{section}{5}
\section{Software Design}
\bigskip

% this first line is probably redundant, but for now im just using it to indicate that there are different types of devices each with their own software environment.
The Akriveia system will feature 3 different devices, namely: the server, beacons, and id tags.
There are two primary software envirnoments: the server is a debian based linux environment including support for raspbian, ubuntu and debian itself.
Additionally, each beacon is an embedded device running without an operating system using the arduino runtime environment.
Each id tag will run under a similar envirnment to the beacons, with different application logic.

\bigskip
\subsection{Akriveia Server}

The Akriveia server is the primary access point for Administrators and First Responders, of whom will be presented a Graphical User Interface through a web server which will serve a static webpage.
The software on this device is written entirely in rust, including the static client side webpage; this is achieved using the relatively bleeding edge rust library called yew, which makes use of rust's LLVM backend to output a static javascript file and adds additional functionality such as html generation which will be used to render the GUI.
This technology is similar to Facebook's React library for javascript.
% TODO make the ref actually point to something valid
Please see the \ref{ui-appendix} for additional details on the UI layout.
\bigskip

\subsubsection{Libraries}
The Akriveia server uses the Actix rust library for its concurrency model, and additionally for its Actix Web submodule for a webserver framework.
Actix makes use of the actor model of concurrency to help alleviate the burden of multithreading in rust. Akriveia makes use of this model to handle multiple beacon connections at once.
The Actix Web library is a web framework which eases the development of typical REST webservers.

% TODO make a reference to the actor concurrency model


\subsubsection{Functionality}
% TODO flesh these out into paras
functions of the server:
	- collecting data from each beacon
	- sending commands to each beacon
	- performing calculations to determine the location of each id tag
		- time of flight
		- rssi
	- serving REST requests from the client webpage
	- serving static html for new client sessions

\subsubsection{Model View Controller}

% TODO format this into a table
User(model)
	- usertype
		- explanation: indicates the type of user, either admin, employee, emergency contact, or first responder
		- type: enum
	- full name
		- explanation: human readable name to identify the user on lists
		- type: string
	- phone number
		- explanation: phone number to contact the user
		- type: formatted string
	- emergency contact
		- explanation: reference to another user as an emergency contact. emergency contacts wont have an employee id or tag id.
		- type: User
	- notes
		- explanation: notes about the user, ie allergies or disabilities, etc
		- type: string
	- employee id
		- explanation: employee id for each user, optional, just for company management
		- type: string
	- id tag id
		- explanation: table key, unique, id for each user.
		- type: 64 bit int
	- last seen/ping timestamp
		- explanation: helps to identify if the id tag data is stale. used to determine if the employee is at work while the disaster occurs.
		- type: unix timestamp

Beacon(model)
	- mac address
		- explanation: unique device identifier for beacon, table key
		- type: string
	- name
		- explanation: human readable string for the device
		- type: string
	- notes
		- explanation: notes for the beacon such as which room the beacon is located at.
		- type: string
	- mapid
		- explanation: reference to a floor/map - this is really just the floor number. Shows the many-to-one relationship between beacons and floors.
		- type: string (same as Map key)


Map(model)
	- name
		- explanation: human readable floor name for gui
		- type: string
	- floor number
		- explanation: table key, unique floor number of the map, this is not an integer to accomodate the possibility of odd floor naming conventions. eg floor 1A, or basement.
		- type: string
	- bitmap data
		- explanation: bitmap for the floor blueprints to view on the gui
		- type: binary


Maps(controller)
	explanation: handles requests for map operations such as to create, update, delete the map instances.

Beacons(controller)
	explanation: handles requests for beacon operations such as to create, update, delete beacon the model instances.

Users(controller)
	explanation: handles requests for user(id tag) operations such as to create, update and delete user instances.

\bigskip

% TODO flesh these out into paras
functions of each beacon
	- sending data to the server
	- responding to commands from the server and acting on them
	- store data on each id tag encountered in an hash table entry - use the mac address as the key
	- store a rolling buffer for each id tag so that data points can be averaged, reducing error
	-

\bigskip

% TODO flesh these out into paras
functions of each id tag
	-




