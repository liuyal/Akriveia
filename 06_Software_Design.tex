

\setcounter{section}{5}
\section{Software Design}
\bigskip

% NOTE: future work would be to build a system to flash new beacons with firmware for the credentials to communicate with the system.
% NOTE: assume beacons have line of sight to eachother. this allows us to make the callibration phase easier by manually measuring the distance between beacons rather than creating a callibration script.

%Libraries/Packages/Frameworks
%Software Stack

% discuss the system at a high level, but leave details to subsections
% breifly reintroduce important topics that the user will need to be reminded of
\subsection{Software Overview}
% TODO revise
The Akriveia system is comprised of 3 different devices, namely: the data processor, beacons, and id tags; this device diversity plays a large role in the software choice for the beacons and id tags.
\smallskip
The Akriveia data processor is the primary access point for Administrators and First Responders, of whom will be presented a Graphical User Interface served by the data processor in addition to handling commands from the client.
The software on the server is written entirely in rust, including the static client side webpage.



\smallskip
The requirements for each device are different, however the id tags and beacons are able to share a similar environment.
As such Akriveia will require two different software environments.

\bigskip
% language choices, formatting, standards, libraries
\subsection{Software Stack}
Akrivia is composed of two primary languages, the server is implemented entirely in rust while the beacons and id tags use C++.

\bigskip
\subsubsection{Software Environments}
The data processing server requires an operating system to perform mundane CPU scheduling for our multithreaded application, wifi drivers, and a proper file system to handle a database.
As such, the linux - and more specifically the debian destribution of linux - was chosen as the servers operating system.

\bigskip
% TODO add a citation for debian stability
Debian is widely praised for its stability, which makes it ideal for a data processing server because Akriveia simply cannot fail as lives depend on it to work.
Debian brings in the aptitude package manager for dependency management, which allows for quick installation of required driver updates, operating system updates, and any additional packages required by Akriveia.
Debian's Aptitude has the advantage compared to other package managers because it removes the need to worry about dependencies, manual installation, and even installation of sub dependencies.
As Debian is a linux distribution, it has widespread hardware adoption across many different hardware architectures - primarily x86-84 and ARM64 - giving Akriveia the flexibility to run almost anywhere with little cost.

\bigskip
Debian is contrasted by another linux distribution called arch linux which uses a different package management paradigm called rolling releases, favoring earlier adoption of newer packages to more quickly introduce features at the cost of stability.
Another alternative to Debian is Windows, however Windows suffers from high power usage, poor package management, and heavy reliance on proprietary software, whereas linux is free of cost, open source, and has a more inclusive licence.

\bigskip
\subsubsection{Software Languages}
% TODO revise with word flow in mind. segmented thoughts in each sentence.
The server is written in rust.
% TODO make the citation work
Rust is a newer language compared to C, it was created in 2006 \cite{rust_graydon_interview} and its first stable version was released in 2015 \cite{rust_releases}, taking inspiration from C++ and Haskell in its design.
Rust is a systems language designed for stability and robustness, it eliminates entire classes of errors using its borrow checking system, a feature unique to this language.
The borrow checking system defines a strict set of rules to follow, and fails to compile if these rules are broken - even if the application is otherwise logically correct.
As a systems language, rust is compiled to binary rather than executed as a script through an interpreter however this does not stop it from being a higher order language.
The borrow checker statically analyzes the code being compiled, and in doing so is able to statically determine when to allocate and more importantly when to free heap allocated memory.
Fundamentally, this means that without a garbage collector and without manually managing memory, rust provides a mechanism to manage memory without a runtime cost while behaving like a garbage collected language providing great flexibility in expressiveness and allowing higher order operations without taking brain resources.

\bigskip
The beacon and id tag are written in C++.
Ideally, the entire project would be written in one languge, however for the case of programming embedded devices, namely arduinos, we have opted to stick wih the standard toolset shipped with the arduino tools.
C++ is specifically designed for embedded programming, allowing use of higher level features when desired while also giving the developer absolute control over how memory is managed.
On arduinos, memory is very limited so control is key to fitting our program within the small space constraints.


\bigskip
\subsubsection{Software Standards}
% TODO references here
Rust code will follow the the rustfmt standard of source layouts, rustfmt is a tool provided by the rust project that automatically formats all source in the standard format. The rust compiler also provides a few warnings for common formatting issues builtin, which will be heeded.

\bigskip
C++ will follow the iso C++ programming standards \cite{cpp_core_guidelines}. The arduino style guide \cite{arduino_style_guide} was also considered, however the rules list they present is incomplete and more ad hoc than a full style guide.

% introduce major libraries that largely dictate how we implement the system.
\subsubsection{Frameworks}
Akriveia's main server framework is called Actix.
Actix is an Actor Framework that follows the Actor Model paradigm of multithreaded workloads.
The Actor Model operates on objects called actors to orchestrate concurrent computations, it does this by treating actors as seperate entities that execute on an event loop on one or more threads, where the event loop simply looks at a list of events generated by actors and executes each event in order as long as it is not blocked by other events.
Events are generated when actors pass messages to eachother, in response the actor that receives the message can modify its local state, create other actors, send messages to other actors, execute arbitrary logic, and finally send a response to the message.
This programming paradigm prevents the need for locks (and by extension, deadlocks) because the message passing mechanism alleviates the need to manually manage concurrency, opting to use atomics instead.
\bigskip
Actix was chosen as the primarily to give the flexibility of multithreading without incurring the typical thought process overhead associated with multithreading. Additionally, as a framework rather than a library, Actix serves as the basis to make REST webservers through the package Actix Web, which is further discussed under the \ref{software_libraries} section.

% discuss which libraries we are using, and how they will be used by us
\subsubsection{Libraries}
\label{software_libraries}

\begin{table}[H]
\centering
\begin{tabular}{ | m{3.25cm} | m{12.5cm} |}
	\hline
	\textbf{Actix Web} & The Actix Web library is a crate that extends the functionality of the Actix framework to create HTTP webservers. \\
	\hline
	\textbf{Yew} & Yew is a crate that extends Rusts ability to compile to Web Assembly by adding the ability to dynamically generate html on the client in response to user events and server responses. This library is very similar to Facebook's React for javascript. \\
	\hline
	\textbf{Serial Port} & Enables serial communication over USB between the server and beacons.\\
	\hline
	\textbf{Rust Standard Libraries} & This is the standard Rust runtime library, and contains many useful boilerplate functions and generic types. This library is included with the basic installation of rustc. \\
	\hline
\end{tabular}
\caption{Akriveia Dependencies - Rust}
\end{table}

\begin{table}[H]
\centering
\begin{tabular}{ | m{3.25cm} | m{12.5cm} |}
	\hline
	\textbf{Arduino Runtime} & This is the standard set of arduino runtime libraries based off of the clib standard library tailored specifically for Atmel Atmega microcontroller chips. \\
	\hline
	% TODO wifi library
\end{tabular}
\caption{Akriveia Dependencies - C++}
\end{table}

\bigskip
% discuss the model view controller paradigm, and its benefits
\subsection{Model-View-Controller}
% TODO glossary of mvc
The model-view-controller(MVC) paradigm is a common method of seperating concerns between modules.
It is an architectural pattern that seperates concerns in a general way, allowing for code reused and improving parallel developement, it is also a common practice which allows others familiar with the paradigm to ramp up quickly on the project.
The model describes data, data manipulation, as well as data storage.
The view simply describes how to view the model.
The controller acts as the means to communicate between the view and model based on data inputs, and is also capable of higher order operations such as between multiple models.

\bigskip
As the Actix Web library is less mature than frameworks in other languages such as Python's Django or Ruby's Rails, Akriveia's data processor will be an implementation of a large portion of manual MVC boilerplate.
This allows for greater flexibility in implementation, at the cost of marginally higher developement time.
A benefit is that MVC does not have a good answer for funcitonality that falls outside of the architectures coverage such as background tasks.
Akriveia is required to maintain communication with many beacons and perform calculations asyncronously from web requests, which is the perfect usecase for the actor framework which fills the void of the model-view-controller.


\bigskip
% discuss how we plan to multithread the data processor. go into slightly more detail about actix actors, and why this
% is necessary
\subsection{Threading Model}
% TODO make a reference to the actor concurrency model

\bigskip
% discuss the data processor subsystems in detail
\subsection{Data Processor Software Architecture}

\bigskip
% discus the purpose of the webserver, and how it will be used to control the other subsystems dictated by user command.
\subsubsection{Webserver Subsystem}
% TODO flesh these out into paras
functions of the data processor:
	- collecting data from each beacon
	- sending commands to each beacon
	- performing calculations to determine the location of each id tag
		- time of flight
			- send a request to id tags, wait for responses and calculate the tof
		- rssi
			- NOTE: only breifly mention rssi for pos
	- serving REST requests from the client webpage
	- serving static html for new client sessions

\bigskip
\subsubsection{Beacon Manager Subsystem}

\bigskip
\subsubsection{Serial Beacon Communication Subsystem}

\bigskip
\subsubsection{UDP Beacon Communication Subsystem}

\bigskip
% discuss database choice - sqlite and why. lead into our models
\subsection{Database details}

\bigskip
% discuss how we plan to represent data in the database, tie back to model-view-controller(specifically models, which
% will implement all of the database queries for the controllers to use.
\subsubsection{Models}

% TODO format this into a table

\begin{table}[H]
\centering
\begin{tabular}{| m{3cm} | m{3cm} | m{9.5cm} |}
	\hline
	\multicolumn{3}{|c|}{User Model} \\
	\hline
	Coordinates & 2D Vector & Last known location of the user within the map. \\
	\hline
	Data Name & Type & Explanation \\
	\hline
	Emergency Contact & ref:User & Reference to another user as an emergency contact. emergency contacts wont have an employee id or tag id. \\
	\hline
	Employee Id & String & Employee id for each user, optional, may be useful for internal housekeeping of the company that purchases Akriveia. \\
	\hline
	Full Name & String & Human readable name to identify the user on lists. \\
	\hline
	Id Tag Number & 64bit Integer & table key, unique, id for each user. \\
	\hline
	Last Seen Timestamp & Unix Timestamp & Helps to identify if the id tag data is stale. Used to determine if the employee is at work while the disaster occurs. \\
	\hline
	Map ID & ref:Map & Last known map the user was located at. \\
	\hline
	Notes & String & Notes about the user, ie allergies or disabilities, etc. \\
	\hline
	Phone Number & String & Phone number to contact the user. \\
	\hline
	User Type & String & Indicates the type of user, either admin, employee, emergency contact, or first responder. \\
	\hline
\end{tabular}
\caption{User Model}
\end{table}

\begin{table}[H]
\centering
\begin{tabular}{ | m{3cm} | m{3cm} | m{9.5cm} |}
	\hline
	\multicolumn{3}{|c|}{Beacon Model} \\
	\hline
	Data Name & Type & Explanation \\
	\hline
	Mac address & String & unique device identifier for beacon, table key.\\
	\hline
	Map id & ref:Map & Reference to a floor/map - this is really just the floor number. Shows the many-to-one relationship between beacons and floors. \\
	\hline
	Name & String & Human readable String for the device. \\
	\hline
	Notes & String & Notes for the beacon such as which room the beacon is located at. \\
	\hline
\end{tabular}
\caption{Beacon Model}
\end{table}


\begin{table}[H]
\centering
\begin{tabular}{ | m{3cm} | m{3cm} | m{9.5cm} |}
	\hline
	\multicolumn{3}{|c|}{Map Model} \\
	\hline
	Data Name & Type & Explanation \\
	\hline
	Name & String & Human readable floor name for gui. \\
	\hline
	Floor number & String & Table key, unique floor number of the map, this is not an integer to accomodate the possibility of odd floor naming conventions. eg floor 1A, or basement. \\
	\hline
	Bitmap data & binary & Bitmap for the floor blueprints to view on the gui. \\
	\hline
\end{tabular}
\caption{Map Model}
\end{table}


% discuss what each controller will do - i dont expect this one to be very meaty
\subsection{Controllers}
Maps(controller)
	explanation: handles requests for map operations such as to create, update, delete the map instances.

Beacons(controller)
	explanation: handles requests for beacon operations such as to create, update, delete beacon the model instances.

Users(controller)
	explanation: handles requests for user(id tag) operations such as to create, update and delete user instances.

% describe how the view will work, but dont go into details about the GUI visuals - leaving that for the ui appendix
% mention browser, js, html, rust -> webassembly compile
\subsection{View}
% TODO make the ref actually point to something valid
Please see the \ref{ui-appendix} for additional details on the UI layout.


% discuss security considerations, talk about features of rust which help us in this regard.
\subsection{Security}


\bigskip

% TODO flesh these out into paras
functions of each beacon
	- sending data to the data processor
	- responding to commands from the data processor and acting on them
	- store data on each id tag encountered in an hash table entry - use the mac address as the key
	- store a rolling buffer for each id tag so that data points can be averaged, reducing error
	-

\bigskip

% TODO flesh these out into paras
functions of each id tag
	-




\pagebreak
\subsection{Software Design Requirements}