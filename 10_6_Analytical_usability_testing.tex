

\subsection{Analytical Usability Testing}
\bigskip
The Akriveia beacon user interface will be tested to determine the state of its usability at various stages of development. The analytical testing phase outlines testing procedure that will be done by the engineering and design team using heuristic usability evaluations. Each evaluator will independently examine the UI and check for compliance and usability. After collecting the results the team will discuss and compile possible solutions to usability issues and generate a list of solutions. Finally, the redesign will be implemented and regression testing will be done. The analytical usability testing will take place during the prototype and final product phases of development as the user interface is well defined during these two stages. The testing procedure will follow the steps described below.

\bigskip



\textbf{Step 1:} Usability Research Data Collection\\
\medskip
The first step is to collect the data generated by the usability test. Each evaluator will perform simple tasks outlined in the testing appendix (section \#). From the procedures performed issues will be highlighted and documented. Each issue will have the following:
\begin{itemize}
\setlength\itemsep{0.1mm}
	\item An issue identification (ID).
	\item Note where it happened (screen, module, UI widget, flow, etc.).
	\item Task the user was engaging in.
	\item Concise description of the issue.
\end{itemize}
Data collected will be shown in a table similar to the table below:

\def\arraystretch{1.5}
\begin{table}[H]
\centering
\begin{tabular}{ | p{0.5cm} | p{2cm}| p{5cm} | p{5cm} | p{0.5cm} | p{0.5cm} | p{0.5cm}|} 
\hline 
ID & Where & Task & Description & P1 & P2 & P3 \\
\hline
1 & Login Page & Login with wrong Password & No error message for wrong user name input & X & - & - \\
\hline
2 & Map View & Click on beacon icon & Beacon info text too small & - & X & X \\
\hline
\end{tabular}
\caption{Usability test results}
\end{table}	



\textbf{Step 2:} Issue prioritization\\
\medskip
Once sufficient testing has been performed by evaluators of the team, issues must be prioritized as time and resources are limited for this project. Each usability issue receives a grade of severity, influenced by factors such as:
\begin{itemize}
\setlength\itemsep{0.1mm}
	\item Task criticality: Impact on user if the task is not accomplished.
	\item Issue frequency: How many times an issue has occurred with various participants.
	\item Issue impact: How much has it impacted the user trying to accomplish the task.
\end{itemize}

\bigskip



\textbf{Step 3:} Solution Generation\\
\medskip
With the combined feedback and evaluations, the engineers at TRIWAVE SYSTEMS will re-evaluate possible UI designs for each usability issue that occurred during testing to determine the best and optional solution. A list of recommendations and solutions will be generated with usability test results. For each design decision several alternative solutions must be generated to include other possible ways to address the issue. 







