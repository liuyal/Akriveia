

\setcounter{section}{1}
\section{Project Overview}
\bigskip
\subsection{System Overview}
\medskip
The Akriveia Beacon indoor locating rescue system combines hardware, electrical, and software systems to detect and locate multiple occupants within a building during an emergency disaster situation. Each individual component of the system is developed separately in the PoC (Proof of Concept) phase, partially integrated in the Prototype phase and fully integrated in the Final Product phase. 

\bigskip
A high-level system overview presents three Locator Beacons, an ID tag, a data processing unit, and a graphical user interface (Figure \ref{sys_arch}). Using the Time-of-Flight principle (ToF) which is a method for measuring distances between transceivers. Based on the time difference between the emission of a signal after being reflected by an object and its return to the sensor, the distance between a beacon and an ID tag can be estimated r\cite{R2-0}. 

\bigskip
The Locator Beacons transmit ultra-wideband (UWB) signals in the frequency of 3.5-6.5GHz to the ID tag to acquire a response. When the response returns back to the Beacon a ToF measurement is acquired. The ToF data captured by ESP32 (MCU) will be forwarded to a portable data processing unit, a Raspberry Pi, via closed Wi-Fi network with User Datagram Protocol (UDP). Then the processing unit will calculate the distance and coordinates of the ID tags using a trilateration algorithm , then the coordinates results are displayed on a GUI for operators.


\medskip
\begin{figure}[H]
\centering
    \includegraphics[scale=0.60]{./images/00_sys_arch.png}
    \caption{High Level System Layout}
    \label{sys_arch}
\end{figure}
\pagebreak

The final product will demonstrate the fully functional indoor rescue system that detects the location of the ID tags and displays it accordingly on a GUI. Here the of ESP32’s WiFi modules for Beacon to DPU communications can be seen (Figure \ref{final}), as the Beacon will communicate via WiFi communication with the data processing unit. The WiFi network will be a closed private network meaning that the network is only share between beacons and the data processing unit to ensure security, reliability and stability. Furthermore, implementation of a RF harvesting circuit for charging the ID Tag device during deep sleep mode will occur throughout this stage. All the components of the systems will be fully integrated as a close-to-production product. Component circuits and {PCB} footprint will be minimized and proper casing will be made to house all electronics. The data processing unit will provide the user with a full GUI to interact with along with fully implemented features such as importable blueprints and system configurations.

\bigskip
\begin{figure}[H]
\centering
    \includegraphics[width=\linewidth]{./images/03_final.png}
    \caption{Final System Block Diagram}
    \label{final}
\end{figure}

\pagebreak
\subsubsection{System Components}
\medskip
The Beacon and ID tags are composed of Decawave DWM1000 \Gls{UWB} module (Figure \ref{dwm_esp} right) and a Espressif ESP32 microcontroller (Figure \ref{dwm_esp} left). The ESP32 contains a Tensilica Xtensa LX6 microprocessor in both dual-core and single-core variations and includes a in-built antenna, power amplifier, low-noise receive amplifier, filters, and power-management modules. The DWM1000 is an IEEE802.15.4-2011 UWB compliant and \Gls{FCC}/\Gls{ETSI} certified wireless transceiver module based on Decawave’s DW1000 IC \cite{R4-2-1}. This module is a combination of DW1000 IC, a built in antenna, power management system, and clock control for simple design integrations. 

\medskip
\begin{figure}[H]
\centering
    \includegraphics[scale=0.75]{./images/dwm_esp.png}
    \caption{Left ESP32, Right DWM1000}
    \label{dwm_esp}
\end{figure}

The Data Processing Unit is a stand alone single board computer (SBC). For the demonstration of this project a Raspberry Pi 3 B+ is used as the DPU since it is an affordable and robust SBC, but the DPU in theory should be any electrical computer device that is capable of running a basic linux operating system; as the software stack is designed to operate on any linux based system. Since the Pi is cheap, portable and designed with an Cortex-A53 processor it meets the minimum requirements for the DPU.

\medskip
\begin{figure}[H]
\centering
    \includegraphics[scale=1]{./images/pi.jpg}
    \caption{Raspberry Pi 3 B+ Model}
    \label{pi}
\end{figure}



\pagebreak
\subsubsection{Trilateration Method}
\medskip
The Akriveia Beacon 2-D indoor localization solution determines the X-Y coordinates of ID tags by trilateration. The method follows a lateration scheme with absolute distances, which uses distance related metrics such as ToF to determine the distance between a sender and a receiver. In the case where there is a single beacon and ID tag, the distance between the two entities can be interpreted as the radius of a circle traced with the beacon centered as shown in figure \ref{tri}. 

\medskip
\begin{figure}[H]
\centering
    \includegraphics[scale=0.55]{./images/Tri.png}
    \caption{Trilateration Diagram}
    \label{tri}
\end{figure}


\medskip
Such a circle takes on a standard mathematical form as shown in equation (1) below, where variables $x_0$ and $y_0$ are the 2-D coordinates of the beacon position relative to its environment, and $r_0$ is the distance between the beacon and the ID tag. Given the 2-D coordinates the three beacons and their individual distance with respect to the ID tag, three circles can be traced to form an intersection point at the location of the ID tag. Hence, three standard form circle equations are generated to form a system of equations with two unknowns as shown in equation (2). The solution to the unknowns, or the intersection coordinates of the ID tag can be obtained by solving the system.


\medskip
\begin{equation}
	(x-x_{1})^2 + (y-y_{1})^2 = r_{0}^2
\end{equation}

\begin{gather}
	(x-x_1)^2 + (y-y_1)^2 = {r_1}^2 \\
	\nonumber (x-x_2)^2 + (y-y_2)^2 = {r_2}^2 \\
	\nonumber (x-x_3)^2 + (y-y_3)^2 = {r_3}^2  
\end{gather}


\pagebreak
\subsection{Project Scope}

\medskip
\textbf{Goal}

\medskip
The goal of the Akriveia Beacon system is to provide accurate and reliable indoor location tracking of trapping personnel during small scale emergency disaster in near real time. The location information will be complied to aid first responders on locating trapped victims during any emergency disaster situations. Therefore, lowering the search and rescue time for first responders and limiting their exposure to potentially dangerous environments, as well as increasing the chances of survival of trapped victims.

\bigskip
\textbf{Justification}

\medskip
Emergency disasters which occur in urban commercial builds creates extremely dangerous and complex environments for first responders to operate in. One of the most important aspects of any search and rescue operations in emergency disasters is the timely locating of trapped victims. By limiting operator exposure to potentially dangerous environments the Akriveia Beacon system minimize the time spent locating victims.

\bigskip
\textbf{Assumptions}

\medskip
The Akriveia Beacon system must maintain critical functions under emergency disaster situations such as fires and low magnitude earthquakes. Each beacon and ID tag must be designed to withstand various environmental factors such as high temperature, fragile structural integrity, and other dangerous factors. The users of the system must have some form of knowledge of the system prior to operating. Lasty, as an indoor tracking tag device the ID tags must be worn at all times by all personnel associated.

\bigskip
\textbf{Deliverables}

\medskip
The Akriveia Beacon is designed as a system of anchor beacons and wearable ID tags composed of Espressif ESP32s MCUs and Decawave DWM1000 UWB transceiver modules. A minimum of three beacons are needed for trilateration to function. A portable data processing unit will also be part of the system and be implemented in Rust to create a layer of interaction between the users and the system. For the purpose of the demonstration a Raspberry Pi will be the DPU. In theory, any single board computer which meets the minimal specification can be used as the DPU.

\bigskip
\textbf{Constraints}

\medskip
The system as a whole must remain operational throughout the event of a disaster as the search and rescue effort will rely on the information generated by the system. Each component of the system will need a battery backup to maintain functionality if the primary power source is unavailable. As an indoor location tracking system a tag device must be worn by personnel being tracked at all times while in proximity of beacons in a building. As a wearable electronics, the size of the ID tags must be optimized in its design so it is ergonomical as an everyday carry. With all wearable electronics power consumption will be another important aspect to focus on. To reduce maintenance costs the device battery must maintain charge over prolonged periods of time or have methods of wireless charging. 



\pagebreak
\subsection{Risks}






\pagebreak
\subsection{Benefits}
\medskip


