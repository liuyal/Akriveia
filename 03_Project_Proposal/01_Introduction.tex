

\setcounter{section}{0}
\section{Introduction}
\bigskip
Over the last couple of decades, urban centers around the world have faced substantial population growth. As a result, the number of large and complex structures in dense urban areas around the world is rapidly increasing. In Canada alone there are approximating 500,000 commercial buildings \cite{R1-1}. A large population combined with massively complex buildings in relatively dense areas leads to higher risk for damage and casualties in the event of a disaster. Due to increased urbanization and complexity of urban structures, search and rescue operations in indoor urban environments face various complications and uncertainties. According to Statistics Canada, an average of 135 fire related deaths occur within commercial structures each year from 2010 to 2014 \cite{R1-2}. In current practices, first responders know little about the situation until arriving on scene. Once responders are on scene, emergency management have to quickly evaluate the situation and plan for appropriate actions \cite{R1-3}.  Assessments of the structure are conducted with readily available blueprints of buildings along with limited information of last known location of possible trapped victims, usually derived from witness reports. Situational data are created dynamically during this process and the actual rescue process heavily depends on the situational awareness of the first line of emergency response operators \cite{R1-4}.

\bigskip
An important issue that must be considered is how emergency first responders should be dispatched inside the building in the event of a disaster in order to minimize search and rescue time as well as limit potential risk to first responders. In order to pinpoint locations of trapped victims quickly and accurately it is critical to have precise location data. Proper emergency planning and organization takes a substantial amount of time, and having additional accurate information on the locations of trapped, incapacitated or immobile personnel would improve first responders situational awareness which would then improve their own safety and possibly greatly increases the victims chances of rescue and survival. As such, the need for a distinct indoor positioning rescue system is crucial in getting fast and reliable information that allows first responders to be dispatched within buildings in the most optimal and efficient manner. 

\bigskip
The Akriveia Beacon by TRIWAVE SYSTEMS focuses on improving the locating and rescue process of personnel trapped in buildings during or after small scale disasters such as fires and low magnitude earthquakes. This is done through a system of Ultra Wide-Band (\Gls{UWB}) Beacons and Identification (\Gls{ID}) tags for accurate, near real-time location of trapped personnel. Each ID tag uses a UWB transceiver module to communicate with the beacon system with similar UWB transceivers. Given the time between sending and receiving transmission data, the distance can be estimated via time of flight. The Beacons will then forward these distance estimations to a data processing unit using a private WiFi network where it will use trilateration methods to calculate the near real time location of each individual ID tag. The system design allows for multiple ID tags as well as more than three anchor beacons to provide more accuracy through redundancy, making it modular and extendable.

\bigskip
The following proposal will describe the details of the system architecture, project scope, risk-benefit analysis, budget and funding, discuss potential markets and competition, and provide an overview of the project planning and scheduling. Furthermore, a brief company profile will be provided to emphasize the skill sets and experience of each of the TRIWAVE SYSTEMS members to showcase the qualification of the engineers involved in developing the Akriveia Beacon System. 
















