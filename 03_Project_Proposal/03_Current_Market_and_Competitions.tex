

\setcounter{section}{2}
\section{Current Market \& Competitions}

%-----Markets-----$
\subsection{Markets}
Currently there are two major markets available for Global Indoor Location (\Gls{GIL}) and Search and Rescue (\Gls{SAR}) equipment. 
The Global Indoor Location market is worth \$ 3.43 billion in 2015 and is projected to reach \$ 29.4 billion in 2022 \cite{R3-1}. This advent 
comes from the increasing amount of smartphone users and ineffective Global Position Systems (\Gls{GPS}) for indoor use. On the other hand, SAR market 
worth is projected to rise from \$ 113.62 billion in 2017 to \$ 125.66 billion by 2022 \cite{R3-2}. Mainly the increased focus on citizen 
safety and rising terroism/insurgency threats supports the upward trend. Suprisingly, UWB is only utilized for GIL market for commercial use in 
warehousing, breweries, factories and other various applications. However, none uses UWB for GIL in emergency situations as SAR equipment. 
Therefore, Triwave Systems uniquely positions itself between two multibillion dollar markets, GIL and SAR, and allowing freedom to compete in both. 

\bigskip

There are multiple Global Indoor Location (GIL) systems available on the market. Each GIL system relies on their own specific technology and has 
their own benefits and cons. Based on the journal \textit{Localization and Positioning Systems for Emergency Respondiers: A Survey}\cite{R3-3}, 
there are five categories or attributes on which each GIL are evaluated: Accuracy, Information Accessibility, System Adaptability, System 
Architecture, System Autonomy and Cost. Of the numerous technologies stated in the journal, only UWB provides the sufficient accuracy and 
reliability needed for indoor tracking for individuals during emergencies. In fact, UWB if optimized properly can provide the necessary $<$1m 
accuracy compared to other technologies. UWB operates on the frequency range from 3.5 to 6.5Ghz band which enables higher accuracy, less interference 
and multipath propagation effect. Bluetooth and Wifi utilize 2.4Ghz which severely reduces its accuracy due to multi-path propagation effect 
and interference. Although 2.4Ghz frequency GIL sysytems are inexpensive and readily available, the inconsistent nature of RSSI causes Wifi and 
Bluetooth GIL systems to be inaccurate \cite{R3-4}. UWB although more costly, can provide the necessary location tracking that our emergency 
tracking system requires. Thus, UWB technology is the most appropriate for our specific scenarios and use cases.

%----Competition-----%
\subsection{Competition}
There are a number of competitors in the market that uses the same UWB technology for indoor localization as Triwave Systems. 
However, the application of their products is mainly associated with tracking equipments and items in warehouse or production facilities, 
which differs from the scope of search and rescue and does not compete directly with our product. The various companies, infsoft or Pozyx, use 
UWB for tracking individuals in warehousing. Tracking individuals or items in non-emergency situations do not necessitate the need for high 
precision localization. As well, use cases where power is out are not considered in the product specifications as vital. UWB applications in non 
life-threatening situations do not demand such strict regulation or procedures compared to SAR equipment. Triwave Systems' Akriveia also has 
backup battery power for the beacons in case of emergencies. Since the ID tags are mainly in deep sleep waiting for user input and the beacons 
emergency broadcast, these ID tags can last several years in deep sleep considering optimal conditions \cite{R3-5}. Another defining feature for 
Akriveia systems is the integrability of the ID tag with the access card. There will be no need for employees to carry two separate cards but 
rather one card with both functions. 

%---update the reference numbers --------------------%
\subsubsection{infsoft}
infsoft is a competitor that provides indoor tracking solution mainly to industrial environments using UWB sensors. 
Similar to Triwave Systems, their tracking system consists of two components: locator tags and locator nodes, for which the locator tags 
can be attached to people or objects and the locator nodes are installed at fixed locations in an infrastructure. Some suggested uses cases 
for their UWB locating systems are forklift localization, route analysis, asset tracking and tugger train management, which are mainly in 
warehouse or production facilties \cite{R3-6}. In applications where locator tags are attached to immobile objects such as asset tracking, 
such a system would be beneficial as it may prevent incorrect inventories. However, in applications where the locator tags are attached to 
people such as search and rescue, privacy becomes an issue as their navigation within the coverage area of the locator nodes are exposed in 
real time. Triwave Systems ensures that user privacy is preserved in all non emergency situations by preventing ID tags to broadcast signals 
which forbids data reception from our beacons. This solution also decreases power consumption of ID tags significantly, hence lengthening 
their battery life as well. 

\subsubsection{Pozyx}
Pozyx is another company that provides tracking solutions using UWB technology for production and other industries. With the same system 
components as of infsoft, the company categorizes its products in two families: Creator and Enterprise. The Creator kit pricing at 
\EUR{1050} consists of 4 tags and 5 anchors is aimed for hobbyist prototyping purposes, whereas the Enterprise kit pricing at \EUR{3900} 
consists of 3 tags and 6 anchors is used in large industrial environments \cite{R3-7}. In terms of accessibility, both product families may 
not be ideal for their dedicated comsumers due to their high price points. Despite difference in use case, Triwave System integrates 
components that are less costly to build ID tags and beacons in order to maintain a widely accessible yet cost effective system. 

\subsubsection{Articles/Journals}
Instead of tags and anchors, the wireless sensor invented by the King Abdullah University of Science and Technology is a handheld portable 
device engineered to detect human body movements such as respiratory chest movements with UWB technology. It is low cost solution with high 
level of accuracy and the ability to penetrate obstacles due to benefits of UWB technology, which is seemingly ideal for search and rescue 
\cite{R3-8}. However, problems arise due to the handheld nature of the device in search and rescue. Firstly, the chance of locating victims
is highly dependent on the coverage area and direction of the search conducted by the device user. This human dependency may pose limitations 
to its effectiveness on search and rescue since it is difficult to have full coverage on a disaster site. Lastly, having an extra device to 
carry can also be an extra burden to first responders who carries gears and equipment during a search operation. Triwave Systems solves these 
problems by having UWB sensors attached to potential victims instead of first responders. Search coverage is also handled by the distribution 
of our beacons and our data processing unit allows first responders to locate victims more effectively without carrying a device into dangerous 
environments.

\bigskip
One article \textit{Indoor localization for evacuation management in emergency scenarios} discusses about using a WiFi system that computes 
indoor localization to the person’s phones \cite{R3-9}. The sensors that detect smoke would send an alarm to the web server which would in turn 
send alerts to Rescuer apps and Worker apps. The phone app would guide the person to exits or provide emergency assistance. However, the authors 
exaggerate the effectiveness and accuracy of RSSI. Thus, the article only provides the concept of an Emergency Indoor Localization system without 
physically implementing it. 
