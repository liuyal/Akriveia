

\setcounter{section}{1}
\section{Design Optimizations}

\bigskip
The Akriveia Beacon system has received major design changes and optimizations in terms of hardware, electrical, and software components. These changes are necessary to ensure the final production ready version cohere to standards and requirements specified in the design requirements, as well as to ensure a quality and reliable product.

\bigskip
Previously Received Signal Strength Indicator (RSSI) with Bluetooth Low Energy (BLE) was used in the proof-of-concept for estimating distance. This yield undesirable ranging results as RSSI is affected by a wide variety of factors such as temperature, humidity, multi-path, and many other environmental factors and uncontrollable RF interference. To ensure ranging accuracy is well within 50 cm as specified in the design requirements, the system's radio technology was changed to time of flight (TOF) associated with ultra-wideband (UWB) for its superior accuracy in distance measurements. Beacon and ID Tag BLE transceivers modules was replaced with DWM1000 modules paired with 3.3V 8MHz Ardunio Mini Pros as the microcontroller unit (MCU). As a result of these changes the ranging capability was improved upto 30m indoors with an accuracy of +/- 25cm. 

\bigskip
WiFi communication has also been adopted in place of the wired serial connections between beacons and data processing server (DPS) to increase scalability and to eliminate wire management issues introduced during installation. This was done by introducing the addition of ESP WiFi modules. As a result each beacon was redesigned to accomuncated the addition of WiFi modules. The MCU communicate with the WiFi modules via hardware serial to be able to receive and send UDP packets from the DPS.

\bigskip
Beacon recovery mechanisms was also designed into the core circuit. A MOSFET trigger switch composed of a N-type MOSFET has been added to enable hardware reboot which can be initiated by server commands in-case of beacon failures. However, since the output voltage from MCUs are 3.3V and the MOSFET trigger needs a higher potential of 5.0V inordered to be triggered, a Bi-directional logic converter consists of a 10K, 100K resistor and a N-type MOSFET has been added into the design. Other design changes and optimization aimed for the production ready version include the addition of rechargeable battery packs equipped with power switches to beacons, removal of breadboards by assembly beacon and ID tag components onto perfboards, and 3D printed cases designed for housing all hardware and electrical components, which can also be mounted.
