

\setcounter{section}{3}
\section{Risk Management}
\bigskip
This section discusses the possible risks associated with design changes of the Akriveia Beacon system, and discuss the possible methods for mitigating these risks. There are three major risk categories associated with design changes, mainly management risks, technological risks, and external risks outside the control of the project team.

\bigskip
Management risks pertain to possibilities of error in company operations due to design changes of the core product. As system design iterates, new materials and components are needed which creates a variety of risks. These include escalating costs, delayed schedule and logistical issues from production, manufacture, and shipping. Delays in project schedule could severely impact development and delivery time of milestones. To avoid delays a purchases must be planned in advance. Sufficient background research are conducted for suppliers  and vendors to ensure reliability. Backup components, vendors and suppliers must be considered.

\bigskip
Technological risks from design changes prevents the system from meeting requirements and performance expectations. These contain component or system failures, software bugs and performance issues. Beacon failures pose the highest risk associated with the system with design changes. There are multiple ways to solve this issue. Currently the prototype design does not include recovery mechanisms, therefore having multiple beacons dispersed throughout the monitored area can ensure redundancy in case of failures. The system needs a minimum of three working beacons within the area, having more beacons ensures that the system remains operational during unexpected situations.To further optimized the failure recovery design, the beacon firmware and server software will be incorporated with recovery functionalities such as reboot or beacon order re-orientation. 

\bigskip
The accuracy of location tracking are also a potential risk to be considered. Inaccurate location could result in misinformation for first responders and hinder their rescue procedures. As more design changes are incorporated into the system, ranging accuracy could be affected by properties such as casing material, casing design, and other environmental factors. This can be mitigated in the installation and calibration process for the system. Each deployment of the Akriveia Beacon system will be calibrated with its environment and complete system test drills. During these process engineers and technicians will be checking for system accuracy and reliability of location to make further. 

\bigskip
The refined prototype also includes many major updates to the core beacon hardware. This includes the additions of ultra-wideband transceiver modules and WiFi capabilities. With the addition of electronic components the impact of external electrical failure is much greater. This is mitigated by incorporating 10000 mAH backup battery packs that will sustain the necessary power level for each beacon for upto 10 additional hours after external electrical failure.

\bigskip
Lastly, external risks may include changes in user market or financial markets, possible patent or litigation issues, and external environmental hazards such as fires or natural disasters. In the final production ready version, the beacons components are encased in a 3D printed casing made with PLA which could sustain structural integrity up to 65 degree Celsius. Which is fairly low considering building fires could reach a temperature of 600 degree Celsius. To meet safety standards of fire safety equipment, the beacon casing must be designed with fire-safe polymers or even coated with fire retardant paint to maintain operation under high temperature environments. The internal components could also be insulated with material such as polybenzimidazole fiber a synthetic fibre which does not exhibit a melting point or light carbon foam. However, with these design changes the project, and manufacture cost will increase as these fire retardant materials mentioned are fairly expensive. 
