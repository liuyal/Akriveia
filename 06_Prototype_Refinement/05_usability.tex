

\setcounter{section}{4}
\section{Human Factors \& Usability}

\bigskip
Hardware components are designed and implemented with usability and serviceability in mind. It is crucial for the product to be intuitive to use since it is to be used in emergency situations. On ID tags, a simple toggle switch is used to activate and deactivate the device. The switch is labelled by text, a simple color scheme, and LEDs indicating on and off state. Access to internal components is restricted to discourage unwanted tampering. For beacons, similar measures with color scheme and LEDS are used to indicate device power state. The internal components will not be permanently inaccessible like ID tags, and will be allowed for maintenance if needed. The beacon casing provides a sliding lid which can be locked into place to prevent tampering, since accessibility will only be given to authorized personnel.  Once unlocked, the lid can slide open to allow easy access to internal circuitry and components. 

\bigskip
Software UI are designed and implemented with the user in mind. The target users, first emergency-responders and adminstators, should be able to use and navigate the software intuitively. 
Entering the URL when connected to the wifi network, the user is greeted with a selection between \textit{Admin} and \textit{First Responder} Login confirmation. This ensures that the 
first responders only view the immediate and important pages instead of any configuration pages. To improve usability, a traditional website layout theme with navigation tabs at the top is used. 
However, first responders do not have access to user, beacon or map list and any configuration settings. Only adminstrators have access to all details of the system including adding and editing
users, beacons and maps. Making easibility a priority, the \textit{Start} and \textit{End} emergency buttons are enlarged and shown at the top of the navigation bar. Users should be able to locate 
the emergency buttons succintly and with ease in times of high stress emergencies.