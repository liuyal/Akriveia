\setcounter{section}{4}
\section{Safety \& Sustainability}
\bigskip
Section Text

\break
\subsection{Safety}
\bigskip
Section Text


\break
\subsection{Sustainability}
\bigskip
The engineers at TRIWAVE SYSTEMS is committed to ensuring the final products is not only functionally effective, but also in compliance with the best environmental sustainability practices. AS such, TRIWAVE SYSTEMS will be dedicated to minimizing impact of the environment by making design choices that are environmentally sustainable. When possible, the Akrivia Beacon will be manufactured with biodegradable and non-toxic materials, as well as materials that are easily recyclable or repurposable to ensure that the waste output is minimal. The material of choice for beacons and ID tag casings will be biodegradable non-toxic Polylactic acid or PLA plastics. PLA is a natural, bio based alternative to petroleum laden ABS and used commonly in 3D printing process. Electronics and circuitry of the product will be designed as small as possible with a minimal footprint. Power sustainability is also considered by incorporating the use of RF harvesting. The need for constant battery replacement is minimized by converting radio frequency power into DC voltage to power devices. This allows the product to need less maintenance and can be sufficient for longer periods of time. Furthermore, by designing a modular system, each component can be replaced and/or upgraded without replacing the entire system. Additional beacon and ID modules can be added and removed with ease. Therefore, lowing the maintenance cost and materials needed.