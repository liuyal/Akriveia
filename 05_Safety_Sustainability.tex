%\documentclass[11pt]{article}
%\usepackage[document]{ragged2e}
%\usepackage[a4paper, total={6.5in, 9.5in}]{geometry}
%\usepackage{graphicx}
%\usepackage{float}
%\usepackage{multirow}
%\usepackage[table, svgnames]{xcolor} 
%\usepackage{array}
%\usepackage{cellspace}
%\usepackage{etoolbox}
%\begin{document}

\setcounter{section}{4}
\section{Safety \& Sustainability}
\bigskip
\subsection{Safety}
\medskip
The Akrivia Beacon system is a combination of electronics, wireless communication, and software systems all functioning together to create a reliable solution. Since the system is designed for disaster related situations, safety is paramount. There are essentially two conditions the system would be under; an idle mode from day to day; and an emergency mode where the system is trigger to transmit location data during a real disaster situation.

\bigskip
During idle mode or normal day to day operations the ID tags would be worn much like an access card. As such, electronic components on the ID tags must agree to standardized safety measurements. During emergency mode under a disaster situation, the beacons are turned on to transmit and receive data from the ID tags. As the associated ID tags will be transmitting ultra-wideband radio frequency both to and from the radio modules which will be worn by a person, the power frequency level must be within limits that are safe for human exposure [safe1]. Additionally, the Akrivia Beacon system will satisfy the following safety requirements:

\bgroup
\def\arraystretch{1.5}
\begin{table}[H]
\centering
\begin{tabular}{ | m{3cm} | m{13cm}| } 
\hline
REQ & C\\ 
\hline
REQ & P\\ 
\hline
REQ & F\\  
\hline
\end{tabular}
\caption{Safety Requirements}
\end{table}	

\break
\subsection{Sustainability}
\bigskip
The engineers at TRIWAVE SYSTEMS is committed to ensuring the final products is not only functionally effective, but also in compliance with the best environmental sustainability practices. As such, TRIWAVE SYSTEMS will be dedicated to minimizing impact of the environment by making design choices that are environmentally sustainable by following the Cradle to Cradle (C2C) standards. C2C refers to the process of development where all components used in manufacturing are able to be brought back into the development cycle [sus1]. By following the C2C Certified standards, the Akrivia Beacon system can be repurposed or recycled as shown below in the figure by EPEA.
\begin{figure}[H]
\centering
    \includegraphics[scale=0.70]{./images/BioTechCycle.png}
    \caption{Biological and Technical C2C cycles [sus2]}
\end{figure}

When possible, the Akrivia Beacon will be manufactured with biodegradable and non-toxic materials, as well as materials that are easily recyclable or repurposable to ensure waste output is minimal. The material of choice for beacons and ID tag casings will be biodegradable non-toxic Polylactic acid or PLA plastics [sus3]. PLA is a natural, bio based alternative to petroleum laden ABS and used commonly in 3D printing process. Electronics and circuitry of the product will be designed with a minimal footprint. Power sustainability is also considered by incorporating the use of RF harvesting. By converting radio frequency power into DC voltage to power devices, the need for constant battery replacement is minimized lowering maintenance costs. The following table includes the sustainability requirements for the system:

\bgroup
\def\arraystretch{1.5}
\begin{table}[H]
\centering
\begin{tabular}{ | m{3cm} | m{13cm}| } 
\hline
REQ & C\\ 
\hline
REQ & P\\ 
\hline
REQ & F\\  
\hline
REQ & F\\ 
\hline
REQ & F\\ 
\hline
\end{tabular}
\caption{Sustainability Requirements}
\end{table}	


%\end{document}