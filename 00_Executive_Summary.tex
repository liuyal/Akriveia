

\section*{Executive Summary}	%For left alignment of title
\bigskip
A major concern during search and rescue operations involving large and complex commercial buildings is locating potential trapped victims. In current conventions there are no reliable methods of accurately locate trapped victims for first responders as commercial building structures are large and complex in design. The Akriveia Beacon by TRIWAVE SYSTEMS focuses on creating an indoor location system with the objective of designing a safe and dependable method of locating victims trapped in complex urban environments during small scale disasters such as fires and low magnitude earthquakes. Such system will allow search and rescue operations to be carried out safely and efficiently, minimizing potential damages and casualties.
\bigskip
Utilizing various hardware, electrical and software components, the engineers at TRIWAVE SYSTEMS is aiming to create a solution that can accurately and reliably to locate and identify trapped victims within complex commercial building structures in near real time. By pinpointing the exact coordinates of any trapped victims associated with an ID tag, the search and rescue time and effort for first responders can be minimized, which is critical in any disaster rescue operations. This is achieved by incorporating a combination of advanced Decawave Ultra-wideband radio modules, espressif ESP32 microcontroller units, data processing units, and reliable trilateration techniques to create a dependable realtime indoor location positioning system.
\bigskip
This project document proposal provides a brief overview of the system architecture, provide a perspective on current markets and potential competitions, discuss project planning, available budget, and funding, as well as an introduction to the development team at TRIWAVE SYSTEMS. The proposal will cover the three phases of product development: Proof-of-concept (PoC) phase, Prototype phase, and Final Product phase.
\bigskip
TRIWAVE SYSTEMS is dedicated to creating a reliable and robust system design to improve disaster search and rescue operations with human safety as the pivotal focus
